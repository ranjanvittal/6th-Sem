\documentclass[12pt]{exam}
\printanswers
\usepackage{amsmath,amssymb,complexity}
\begin{document}

\hrule
\vspace{3mm}
\noindent
{\sf IITM-CS6840-2015 : Advanced Complexity Theory  \hfill Given on: Jan 27, 07:30am}
\vspace{3mm} \\
\noindent
{\sf Problem Set \#1 \hfill Due on : Feb 04, 07:00am}
\vspace{3mm}
\hrule

\begin{questions}

%\question[5]
%Show that computing the permanent of matrices with integer entries is in %$\FP^{\#\SAT}$.

%\begin{solution}
%Write Solutions like this.
%\end{solution}

\question[10]
Show that $\#\SAT$ remains $\#\P$-complete even if we restrict the input formulas to be of the form 2-CNF.

\question[10]
Show that is there is a polynomial time algorithm that approximates the number of
simple cycles in a given graph within a factor of $\frac{1}{2}$, then $\P = \NP$.
\begin{solution}
Consider Hamiltonian cycle problem.\\
Input is G a directed graph. Now introduce $2*n^2+2$ vertices for every vertex in G
such that there are $2^{n^2}$ paths between a first vertex and the last vertex.
The first vertex is the sink that is all the in edges to a vertex v comes into
the first vertex and the last vertex in the source where the edge starts from
this vertex. The $2^{n^2}$ construction is similar to the one we saw in class.
Now if there is a Hamiltonian cycle in this graph then this graph has atleast
$2^{n^2*n}$ that is atleast $2^{n^3}$ cycles in this graph.\\

$n_{p_k} \le n_{c_k}*(n!)$ which implies $\Sigma n_{p_k} \le 2^n*(n!)$.\\
Typically every permutation in a graph can correspond to a cycle.
Therefore total number of cycles in a graph is bounded by $2^n*(n!)$.
Now if there doesnot exist a single Hamiltonian cycle in the original graph, then
each of the above cycle can contribute to atmost $2^{n^2*(n-1)}$ number of cycles
which is equal to $2^{n^3 - n^2}$ number of cycles. Therefore total number of cycles
would atmost be $2^{n^3 - n^2 + n + nlog n}$ (n log n comes in because $n! \le n^n$).
Asymptotically $c*2^{n^3 - n^2 + n + nlog n} < 2^{n^3}/c$ for any constant c.
Hence if there exists any
constant approximation solution for number of simple cycles in a graph, then given
any graph G we can apply the above transformation to get G' and feed it to this
simple cycle approximation machine. If it says that there are more than $2^{n^3}/c$
cycles then this immplies there definitely exists a Hamiltonian cycle by the above proof.
And if it doesnt reach this much then it definitely means that there doesnt exist
as many cycles in the original graph meaning there doesnot exist a hamiltonian cycle.
Hence using the approximation algorithm we can deterministically tell whether
any input graph has a hamiltonian cycle or not. Hence $P = NP$ if there
exists such a polynomial time approximation algorithm.
\end{solution}
\question[10]
Define $Gap\P$ to be the set of functions $f : \{0,1\}^* \to \mathbb{Z}$ such that there is a choice Turing machine for which :
\[ \forall x, f(x) = \#\textrm{{\sc accept}$_M(x)$} - \#\textrm{{\sc reject}$_M(x)$} \]
Show that :
\begin{parts}
\part Argue that $\#\P$ is contained in $Gap\P$.
\part Show that $Gap\P = \#\P - \#\P = \#\P - \FP = \FP - \#\P$, where minus signs refer to difference of functions belonging to the corresponding classes.
\end{parts}
\begin{solution}
\begin{parts}
\part Consider the machine M which recognizes a function $f \in \#\P$.
Let M have a accepting paths and r rejecting paths.
Now consider another machine M' which has the exact same description as M
but whenever it reaches a reject it introduces 1 accept and 1 reject. So this
new machine will have a+r accepting paths and r rejecting paths.
The number of accept - number of reject in this machine is equal to
the number of accept in the previous machine.
Hence the $Gap\P$ function which is recognized by this machine
is equal to f. Hence \#P is contained in $Gap\P$.
\part
Let us prove first that any function of the form $\#\P - \#\P$
is recognizable by a $Gap\P$ machine. Consider $f_1$ recognized by $M_1$ and
and $f_2$ recognized by $M_2$ (both \#P functions). Let us make
a $Gap\P$ machine M recognizing $f_1 - f_2$. Let on input
x $M_1$ has $a_1$ accepting paths and $r_1$ reject paths and
and $M_2$ has $a_2$ accepting paths and $r_2$ reject paths.
M makes an intial branch into two nodes. From the
first node we start simulating $M_1$ and from the second we start
simulating $M_2$ except that in $M_2$ we make all accept go to reject
and all reject go to accept. The number of accepts from the first node
is $a_1$ and from the second node is $r_2$ and number of rejects
from the first node is $r_1$ and from the second node is $a_2$. Therefore
$Gap\P$ function for this input will be $a_1 - r_1 - (a_2 - r_2)$ = $f_1 - f_2$.
Hence any $\#\P - \#\P$ can be represented by a $Gap\P$ function.
Hence $\#\P - \#\P \subseteq Gap\P$.\\
$FP - \#P \subseteq \#P - \#P$ because $FP \subseteq \#P$.\\

Lets prove that any $Gap\P$ function h recognized by some
machine M can be recognized by $f-g$ where $f \in FP$ and $g \in \#\P$.\\
The machine M is bounded by some polynomial time $p(n)$.
Let on some input x it has a accept states and r reject states.
Now consider the current modification to M. Whenever it reaches accept
at level l from root, we introduce $p(n)-l+1$ new levels
and accept the first $2^{p(n)-l}+1$ and reject the rest of the nodes
that is the rest of the $2^{p(n)-l}-1$ nodes and similarly for
a reject node l level deep introduce $2^{p(n)-l}+1$ reject nodes
and $2^{p(n)-l}-1$ accept nodes (introducing so many accept nodes
can be thought of as simulating $2^{p(n)-l}+1$ which is in FP and FP is a
subset of \#P and hence simulate it using a corresponding \#P machine right
at that node). Let this machine be M'
Now every accept in the original machine introduces
$+2$ to $Gap\P$ count and every reject node gives
a $-2$ to $Gap\P$. Hence total = 2*h(x).\\
We can note that accept + reject = $2^{p(n)+1}$
because all of them reach the same level finally.
Therefore 2*h(x) = $acc - rej$ of curent machine = $total - 2*rej$
of current machine where total = $acc + rej$ of current machine.
Therefore h(x) = $total/2 - rej$ = $2^{p(n)} - rej$. Now consider
machine $M''$ which accepts whenever $M'$ rejects and rejects
whenever $M'$ accepts. Therefore the \#P function(g) recoognized
by $M''$ is the number of rejects in $M'$. Therefore
h(x) = $2^{p(n)} - g(x)$. Let $ f = 2^{p(n)}$. We know that
f is polynomially computable and g belongs to \#P.
Therefore $h = f - g$. Hence $Gap\P \subseteq FP - \#\P$.
Hence $Gap\P = \#\P - \#\P = FP - \#\P$. As $Gap\P = \#\P - \#\P$,
it is trivial that it is closed under negation which implies
$FP - \#P$ is closed under negation. Given $h \in Gap\P$,
$-h \in Gap\P$ which implies $\exists f \in FP$ and $g \in \#\P$
such that $f-g = -h$ which implies $g-f = h$. Hence for any $h \in Gap\P$,
$\exists$ $g \in \#\P$ and $f \in FP$ such that $g-f = h$ which implies
$Gap\P \subseteq \#\P-FP$. But as $Gap\P = \#\P - \#\P$, and trivially
$\#\P - FP \subseteq \#\P - \#\P$ because $FP \subseteq \#\P$,
$Gap\P = \#P - FP$.\\
Therefore, $Gap\P = \#\P - \#\P = \#\P - \FP = \FP - \#\P$.

\end{parts}
\end{solution}
\question[20]
A language $L$ is in $MP$ if there exists a function $f$ in $\#\P$ and a function
$g \in \FP$ (called a bit selection function) such that for all $x$, $x$ is in $L$ if and only if there is
a $1$ at position $g(x)$ in the binary representation of $f(x)$. Show that
\begin{parts}
\part $\PP^{\parity\P} \subseteq MP$.
\part $MP$ is closed under complementation.
\part $\P^{\PP} = \P^{\#\P} = \P^{MP}$.
\part For every $L \in MP$ there is a $\#\P$ function $f$ such that for all $x$, $|f(x)|$ is odd, and $x \in L$ if and only if the middle bit of $f(x)$ is 1.
\end{parts}
Thus conclude that access to the middle bit of a $\#\P$ function is more powerful than the MSBs and the LSBs.
\begin{solution}
\begin{parts}
\part
\part
Given any $f \in \#P$, we can make $f+1$ in the following manner:\\
From the root simply branch out to left and right where in the left
you run NP machine corresponding to f and on the right you simply accept.\\

Given any $f \in \#P$, we can make $2^{p(n)+g(x)+2}-f$ where p(n) is
the running time of the NP machine corresponding to f in the following manner:\\
Whenever it accepts branch out till you reach level p(n)+2 and reject only in
the left most path and if it rejects, branch out p(n)+2 and accept all paths.
Here it is trivial to see that number of accepts of the original machine
is equal to the number of rejects of this machine. Hence number
of accepts of this machine = $total - \#rejects = 2^{p(n)+g(x)+2}-f$.\\

Given any f make f' = f+1 and then make $f'' = 2^{p(n)+g(x)+2}-f' =  2^{p(n)+g(x)+2}-1 -f$.
Hence for any f, f''(x) will give the 1's compliment of f(x) with p(n)+2
bits. Now given any $L(f, g) \in MP$, L(f'', g) represents
the compliment of L(f, g) because all the bits are flipped in the above
translation and hence if L(f, g) gives 1, L(f'', g) will give
0 and when L(f, g) gives 0, L(f'', g) gives 1. Hence MP is closed
under complimentation.
\part
$P^{PP} \subseteq P^{\#P}$.\\
This is trivial because we can simply ask for the MSB of the
answer(\#P) got from the Non deterministic underlying machine used
for PP.\\
$P^{\#P} \subseteq P^{PP}$.\\
Using PP we can first find out the first bit of \#P.
Now for finding the second bit :\\
Let first bit be x. Let original NDTM be M.
For a new M', have a root whose left child runs M and
right child runs a trivial machine which simply
accepts $2^{p(n)}-x10000...$ 0's occurring p(n)-1 times.
Basically if PP returns 1 it means the second bit is 1 and if PP returns
0 then the second bit is 1. In a similar fashion ($2^{p(n)} - xy10000....$)
we can find the third bit and in general any kth bit of \#P and hence
finally arrive at the whole of \#P. Hence instead of querying \#P
we can find out all the bits by querying PP machine
polynomial number of times.\\
$P^{MP} \subseteq P^{\#P}$, this is trivial because we can query the
g(x) th bit after finding the \#P of the f of MP.\\
$P^{\#P} \subseteq P^{MP}$. We can encode g(x) = 1 and query it
to \#P to get the first bit and similarly do it till g(x) = MSB for the same f
to get all the bits of the \#P function f.\\

Hence $\P^{\PP} = \P^{\#\P} = \P^{MP}$.
\part
Given any NDTM M with \#P function f and running time p(n) and a polynomially
bounded g(x),  consider M' such that it branches to two children where on the
left it simply runs M and on the right it runs a trivial machine which reaches
level p(n)+g(x) and accepts
all paths in this right node branch. Now it is easy to see that this machine
accepts $2^{2*(p(n)+g(x))} + f$. Hence given any NDTM M with \#P function f
then we have another NDTM M' such that its \#P function(f') is
$2^{2*(p(n)+g(x))} + f$. The number of bits in f' is odd because leading bit
is at position 2*(p(n)+g(x)).\\

Given any function f in \#P recognized by M and a polynomially bounded
function h consider M', which branches till the h levels and runs M
on each of the $2^h$ paths created. It is easy to see that the function
accepted here is $f(x)*2^{h(x)}$.\\

Let the \#P function of the MP be f with running time p(n)
and bit selecting function be g. Let \#P function recognizing f be M.
Now first transform the machine M to M' which accepts a total of
$2^{2*(p(n)+g(x))} + f$ paths(f'). Number of bits in f'
for any given x is 2*(p(n)+g(x))+1. Now if we multiply this f' by $2^{2*p(n)+2}$
to get f'' which as already seen can be simulated by a \#P function
as already seen above, then it is easy to see that the middle bit
of this f'' and the ${g(x)}^{th}$ bit of f are the same.\\

Therefore given any $L \in MP$, there exists a machine which accepts or reject
based on the middle bit of the number of accepting paths which accepts
L.\\

Hence it is easy to see that middle bit is more powerful than the MSB and
LSB as both are subsets of MP and all of MP is simulatable by the middle bit
guesser.
\end{parts}
\end{solution}
\end{questions}
\end{document}
