\documentclass[solution,addpoints,12pt]{exam}
\printanswers
\usepackage{amsmath,amssymb}
\usepackage[T1]{fontenc}
\begin{document}
\hrule
\vspace{3mm}
\noindent
{\sf IITM-CS6840-2015 : Advanced Algorithms  \hfill Given on: Feb 1}
\vspace{3mm} \\
\noindent
{\sf Problem Set \#1 \hfill Due on : Feb 15}
\vspace{3mm}
\hrule

\begin{questions}
\question
\begin{parts}
\part
Given a graph G, do BFS starting from any node. Let the root be coloured
black and all its children be coloured red. Basically any node in 2*i+1
level is coloured black and any node in 2*i+2 level is coloured red. Now while
doing bfs, if we encounter a node which initially occurred at an odd level
occurring at an even level or vice versa, this implies there exists an odd cycle
in the graph and the graph cannot be $2-coloured$. Otherwise if we go on
colouring in this manner we would end up colouring all the vertices either
red or black and no black and red vertices will have edges between them.
\part
The decision version is as follows :\\
Can you colour the graph using k colours with no two sharing an edge
having the sme colour ? is the decision version of the same problem.\\

Assuming the Graph colouring problem is polynomial time solvable using
Algorithm A:\\
Given any G, k forward G to A. It would gives us a number k' denoting the
minimum number of colours used to colour the graph. If $k < k'$ then
definitely G is not $k-colourable$ while if $ k \ge k'$ then the graph
can be coloured using k colours. Hence decision version is solvable
if Graph colouring is.\\

Assuming decision version is solvable using algorithm A in polynomial:\\
The graph can definitely be coloured with n colours
where n denotes the number of vertices that is every vertex
having a different colour. Now given graph G, ask decision
version (G, 0), (G, 1), (G, 2) so on till (G, n). Now till some
$k-1$ the algorithm A would say no and from k it would start saying yes.
This means that the graph was not $k-1$ colourable while it is indeed
$k-colourable$. Hence the minimum number of colours required to colour the
graph is k. The running time of this machine is polynomial composed
polynomial which is still polynomial hence the
Graph colouring problem is solvable in polynomial time if decision version is.\\
\part
It is given that $3-colouring$ problem is $NP-Complete$. Consider the
following reduction from $3-colouring$ to decision version :\\
Given Input G, give input as (G, 3) to the decision version
the decision version will say yes iff there exists a $3-colouring$ possible.
Hence $3-colouring \le_m decision$ version. Hence the decision version is $NP-Complete$.
\part
Consider a valid $3-colouring$ c of the graph G.
Assuming RED is coloured with red, if one of vertices $x_i$ or $\overline{x_i}$
are coloured red then the colouring rule is not satisfied as RED has an edge to
all these variables. Hence each other variable is either coloured either
true or false. Now both $x_i$ and $\overline{x_i}$ cannot be coloured
the same colour because they share an edge between them. Hence one of them
will be coloured true and the other false.\\

Given any satisfying assignment to the problem, colour $x_i$ to be false
if in the satisfying assignment it indeeds turns out to be so and colour
$\overline{x_i}$ to be true and do similar if $x_i$ is true. Repeat this
for all the vertices. Hence this will produce a valid 3 colouring
between containing only the literal edges as 3 colouring condition will
be satisfied within the literal edges.
\part
Let vertex connected to x be $v_1$, connected to y be $v_2$, connected to
$v_1$ and $v_2$ be $v_3$, connected to $v_3$ be $v_4$ and that connected
to TRUE and $v_4$ be $v_5$.\\
Assuming all x, y, z are coloured false, then
$v_5$ has to be coloured red which implies $v_4$ has to be coloured
false. Now none of $v_1$ or $v_2$ or $v_3$ can be coloured false because
one of each's neighbours are coloured false. As these 3 form
a triangle and one colour is not usable, this widget is not
3 colourable.\\
Assuming z is coloured true :\\
If c(x) != c(y) (c denotes colour), then colour $v_5$ red,
$v_4$ false and $v_3$ red. Now we know x and y can be either coloured true or false.
then colour $v_1$ as compliment of
x (ie) if x is true colour $v_1$ false and so on and colour $v_2$ as
compliment of y. Now this widget is completely coloured using 3 colours.\\
If c(x) = c(y), colour $v_2$ to be compliment of c(y), colour $v_1$ red,
colour $v_3$ same as y and colour $v_4$ red and $v_5$ false.
Hence in this case it is 3 colourable.
Assuming exactly one of x or y is coloured true and z is coloured false:\\
we can easily check that the following strategy works :
(r denotes red, f denotes false and t denotes true)
c($v_5$) = r\\
c($v_4$) = f\\
c($v_3$) = r\\
c($v_2$) = $\overline{c(y)}$\\
c($v_1$) = $\overline{c(x)}$\\
Assuming both are coloured true,\\
c($v_5$) = r\\
c($v_4$) = f\\
c($v_3$) = t\\
c($v_2$) = f\\
c($v_1$) = r\\
Hence the widget is $3-colorable$ if and only if at least
one of x,y, or z is coloured c(TRUE).
\part
Given that there exists a satisfying assignment accordingly colour
all the vertices as indicated in part d and use the schema in part e
to colour all the clause edges. Due to the valid consistency check we
are guaranteed to have a proper 3 colouring overall.\\
Given that there exists a valid $3-colouring$, we
already know that the literals will be coloured either
true or false and that the compliment colour is
also appropriate and that a clause widget
is 3 colourable iff atleast one of the literals in it is
true. Hence assign the variables as the colour suggests and
we are assured to end up with a valid satisfying assignment.\\
Hence by the above proof we have shown that $3-SAT \le_m 3-colouring$
and $3-SAT$ is $NP-Complete$. Hence $3-colouring$ is $NP-hard$.
There is a trivial $NP$ machine to do this problem as verifying
$3-colouring$ is in P. Hence $3-colouring$ is $NP-complete$.
\end{parts}
\question
\begin{parts}
\part
Case 1: When zero end points are specified :\\
First obtain the minimum spanning tree of G(G').
$MST \le Opt$\\
There would be even number of odd degree vertices in MST of G.
Let their number be 2*k. Now consider the optimal hamitonian path
along these odd degree vertices. It is evident that this cost
is definitely less than Opt due to the metric property.
Therefore $Opt_{odd} \le Opt$.\\
The number of edges in that path will be $2*k-1$.\\
The path is composed of a $k-1$ matching and a k matching.
Hence the cost of this path is atleast twice the mincost
matching with $k-1$ edges. This implies
($min_{k-1}$ denotes mincost $k-1$ matching)
$Opt_{odd} \ge 2*min_{k-1} \implies min_{k-1} \le Opt_{odd}/2 \le Opt/2$.\\
Let us find the mincost matching with $k-1$ edges in the following manner :\\
\begin{verbatim}
min = sum over all the edges
for each subset of 2k-2 of the 2k odd vertices :
    min = minimum(mincost maximum matching over these 2k-2 vertices, min)
\end{verbatim}
The running time of the above algorithm is (2k choose 2)*p(n)
where p(n) which is the running time of mincost maximum
matching. It is easy to note that the algorithm correctly
finds the minimum matching with $k-1$ edges and runs in
polynomial time. Now add the edges of this matching to G'.
Now we have a connected graph where each vertex has even
degree except two vertices. These two vertices are the ones
which are left over in the matching. Hence there exists an eulerian
trail covering all these vertices starting from one of the odd
degree vertices. The cost of this eulerian trail is cost(G') + $min_{k-1}$.
Now by repeatedly short circuiting in this trail we will be able
to get a hamiltonian path due to the metric property and we are still
ensured that the cost doesnt increase. Therefore cost
of the obtained path $is \le cost(G') + min_{k-1} \le Opt + Opt/2 = 3/2*Opt$.
Hence we obtained an answer atmost 3/2 times optimum.\\

Case 2 : When one end point is specified :\\
Let specified vertex be v. Now consider the minimum
spanning tree of the subgraph containing the other
$n-1$ vertices. Now add the least edge from v to this
spanning tree to obtain a tree G'. Consider the optimal
path which starts from v. The cost of this path is
definitely is atmost the cost of G'. Hence $Opt \le G'$.\\

Now there will be some even number of odd degree vertices in G'
which includes v, let them be 2k. Let optimum hamiltonian
path across these vertices starting from v be $Opt'$.
Clearly $Opt' \le Opt$.
The optimal among these vertices consists of a k matching
which contains v
and a $k-1$ matching not containing v
. Now find mincost maximum matching
of the other $2k-1$ vertices. This will give us $k-1$ edges
which cover the $2k-2$ vertices. The cost of these edges
is lesser than both the k matching and the $k-1$ matching
described above. Therefore $2*min_{k-1} \le Opt' \le Opt$.
Now add these edges to G' and then we get a graph with all vertices
having even degree except v and another vertex. Now start from
v and complete the tour by short circuiting the eulerian trail
we get. The cost of this $is \le cost(G') + min_{k-1} \le Opt + Opt/2$.
The cost of this tour is less than $3/2*Opt$. Hence a 3/2 factor is obtained.
\part
Let s and t be the terminal vertices given.
Find the minimum spanning tree of the graph G say G'.
Clearly $Opt \le c(G')$.\\
Now consider all the odd degree vertices and even degree end points(either s or t)
in a set S. Now find the mincost maximum
matching of S. Let this matching be G''. Now if we add G' and G''
all vertices will have degree 2 except the two endpoints. Now
we can have an eulerian trail starting from s and ending at t. If we repeatedly
short circuit we would end up having a hamiltonian path with atmost
cost of c(G') + c(G'').\\
Assume that the optimum path is H.\\
As we have seen before $c(G'') \le TSP_{opt}/2$.
Now H and the edge $s-t$ form a cycle with cost atmost $TSP_{opt}$.\\
Therefore $c(G'') \le (c(H) + c(s-t))/2 \implies 2*c(G'') \le c(H) + c(s-t)$.\\

Property : The mincost maximum matching of $any$ subset of vertices
of any graph Q will have a cost of atmost half that of the $TSP_{opt}$.
Now we already know by the double touring mechanism that
$TSP_{opt} \le 2*c(MST)$. Therefore mincost maximum matching of any subset
of the graph Q is less than the cost of MST given metric constraints.\\

Consider $G''' = G' - (s-t)path$ in G'. Degree of some vertices
other than the endpoints have been reduced by 2. This implies
parity of none of the vertices change except that of the endpoints.
The parity of both the endpoints change. Hence it is easy to note
that the odd degree vertices here is exactly same as S.
Consider any forest T in G'''. The mincost maximum
matching of odd degree vertices in T is less than the cost of T due
to the property we see above. Hence consider the union of all mincost
maximum matchings of all forests in G''' which would be
less than c(G'''). Clearly
the sum over all these costs is less than the mincost maximum
matching of S. Therefore $c(G'') \le c(G''')$.\\
$c(G''') = c(G') - c(s-t)$ path. But $c(H) \ge c(G')$
and c($s-t$) path $is \le c(s-t)$ edge because of the metric
property.\\
Therefore $c(G''') \le c(H) - c(s-t) \implies c(G'') \le c(H) - c(s-t)$.\\

Therefore $2*c(G'') + c(G') \le c(H) + c(s-t) + c(H) - c(s-t) \implies
3*c(G'') \le 2*Opt$.\\
Hence $c(G') + c(G'') \le Opt + 2*Opt/3 = 5/3*Opt$.\\
Therefore 5/3 factor is proved.
\end{parts}

\question
\begin{parts}
\part
Assuming that V = {1...n} and V is a DAG,
construct the graph G' =(V',E'), where
V' = ${x_0...x_n} \cup {y_0...y_n}$,\\
E' = $\{(x_0, x_i) : i \in V \} \cup \{(y_i,y_0) : i \in V\} \cup \{
(x_i, y_j) : (i,j) \in E\}$.\\
Let all these edges have a weight 1.
Now let us prove that the max flow from $x_0$ to $y_0$ is k iff the minimum
path cover is $n-k$.\\
Consider a valid flow from $x_0$ to $y_0$. Now include
all the edges chosen in the matching. That is if $x_i$ and $y_j$ are
connected, choose edge $i-j$. It is given that the given graph is a DAG,
hence it cannot have any directed cycles. If a directed non
cycle has to be an undirected cycle then it implies atleast one of the
vertices should have two edges coming from it. But this is not
possible in the set of edges which were chosen by us as it is a
matching. The chosen edges cannot form a tree $non-path$ because
in a tree, two edges cannot emanate from the same vertex. Hence the
edges which we choose definitely form a disjoint
set of paths. Let us assume that t paths are formed
from the set of k edges. Each path covers one more
vertex than the number of edges in that path. Therefore
the number of vertices that are covered by the t vertex disjoined paths are
$n-k+t$(it is easy to see that this is not
more than n). Now $t-k$ vertices are uncovered. Cover all of them with path
length 0 paths. Hence we finally have $n-t+t-k = n-k$ number of paths covering
all the vertices. Therefore given any flow of k from $x_0$ to $y_0$ then
there exists a path cover of size $n-k$.\\

Consider the minimum path cover of all the vertices of size k.
In any of the paths the number of vertices covered is 1 more
than the number of edges in that path. The total number
of vertices covered is n then the number of edges chosen are $n-k$.
Now add all the edges in the cover in the bipartite graph created above.
As in disjoint path covers, no vertex is a source twice or sink
twice, the above forms a valid flow with value $n-k$. Therefore
the minimum path cover has a flow layout corresponding to it and
every flow corresponds to one path cover. As the flow
increases the path cover size decreases and hence the max flow
corresponds to the minimum path cover.\\

\part
Consider the trivial graph where $0 \leftarrow 1$
and $1 \leftarrow 0$. Now by the graph we made above, the
max flow would turn out to be 2 and the answer we get is 0. But the minimum
path cover is 1. Hence the above algorithm works correctly only for DAGs.
\end{parts}
\question
\begin{parts}
\part
Set cover doesnot have c ln n approximation for all constant c
unless P = NP.\\
Given any weighted set cover problem in the following
manner :\\
S = universal set of all elements to be covered($e_i$).\\
$S_i$ = some subset of S.\\
$W_i$ = weight of $S_i$.\\
Transform it into a $node-weighted$ steiner
tree problem as follows :
\begin{itemize}
\item Let all the $S_i$ denote vertices $V_i$.
\item Let all the $e_i$ denote vertices $v_i$.
\item Now $v_i$ is connected to $V_i$ iff $e_i \in S_j$.
All $V_i$ are connected to each other.
\item All the edges have weight zero.
Weights of all the $V_i$ are the weights of the corresponding $S_i$
and the weights of each of the $v_i$ is set as 0.
\item $v_i$ denotes the set of required vertices in the steiner
tree.
\end{itemize}
Note : Given any subset of $V_i$ which cover
all $v_i$, we can easily get a spanning tree out of it.\\

Consider any solution of the above steiner tree problem.\\
The solution would contain a set of $V_i$ and all $v_i$.
As all the edges have zero weight the cost
of the edges contribute zero and hence the cost
of the steiner tree is exactly the cost of $V_i$ chosen.
Now choose all the sets $S_i$ which correspond to
$V_i$ chosen above. Because the tree chosen is a
valid steiner tree it has to have edges to all the $v_i$
which implies $S_i$ got covers all $e_i$ and the $S_i$ has the
same cost.\\
Similarly we can show that for every solution of
the Set cover there is an equivalent steiner tree solution
(Pick all $V_i$ corresponding to $S_i$ chosen). Hence
the above is a valid reduction from Set cover to $node-weighted$
steiner tree and preserves the cost. Therefore if
$node-weighted$ steiner tree has a c ln n approximation
for all constants c, then Set cover has a c ln n
approximation for all c which would imply P = NP.\\
Therefore there exists c such that a $c$ ln n
approximation doesnt exist for $node-weighted$ steiner
tree problem unless $P = NP$.
\part
\end{parts}
\end{questions}
\end{document}
