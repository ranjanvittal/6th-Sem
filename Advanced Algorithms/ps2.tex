\documentclass[solution,addpoints,12pt]{exam}
\printanswers
\usepackage{amsmath,amssymb}
\usepackage[T1]{fontenc}
\begin{document}
\hrule
\vspace{3mm}
\noindent
{\sf IITM-CS6840-2015 : Advanced Algorithms  \hfill Given on: Feb 1}
\vspace{3mm} \\
\noindent
{\sf Problem Set \#1 \hfill Due on : Feb 15}
\vspace{3mm}
\hrule

\begin{questions}
\question
\begin{parts}
\part
Let the graph G be coloured using colours A, B, C, D.
Let D be the colour used to colour maximum number of
the vertices. Therefore $D \ge n/4$ (PHP). Consider
all the vertices coloured by $A \cup B \cup C$, say set S.
It is easy to note that the cardinality of this set
is less than 3*n/4. Any edge having a vertex coloured
with D, has the other vertex coloured with a different
colour (Graph is coloured properly). Therefore
S forms a vertex cover of G as any edge has atleast one
endpoint in S.
\part
A four clique is a graph which is 4 colourable
and planar and has minimum vertex cover equal to 3.
\part
We know that any optimum LP solution to vertex
cover is half integral.
Consider an optimum LP solution to graph G for the minimum
vertex cover. Let $a_1, b_1, c_1, d_1$ be the
vertices which are coloured by A, B, C, D respectively
and are chosen as 1 by the LP and their union be S.
Let |S| = k.
Similarly define $a_{1/2}, b_{1/2}, c_{1/2}, d_{1/2}$ whose union is S'.
Let |S'| = t.
We know that $k+t/2 \le mvc \le k+t$.
(mvc is size of minimum vertex cover).\\
Wlog $d_{1/2}$ be the one which is the maximum among
$a_{1/2}, b_{1/2}, c_{1/2}, d_{1/2}$. Now
we can easily see that any edge which has a vertex from $d_{1/2}$ as
one end point has the other endpoint either in S or S'.
Now consider the set $S \cup {S' - d_{1/2}}$ say S''. Any vertex
in $d_{1/2}$ has its other end in S'' because two vertices
in $d_{1/2}$ donot share an edge because of proper colouring.
Therefore S'' is a valid vertex cover. The size of S''
is atmost $k + 3*t/4$. Therefore $(k+ t/2)*(3/2) \ge k+ 3*t/4$
Therefore $(3/2)*mvc \ge |S''|$. Therefore we have a 3/2 approximation
for vertex cover when the given graph has a 4 colouring given.
\end{parts}
\question
\begin{parts}
\part
Case 1: When zero end points are specified :\\
First obtain the minimum spanning tree of G(G').
$MST \le Opt$\\
There would be even number of odd degree vertices in MST of G.
Let their number be 2*k. Now consider the optimal hamitonian path
along these odd degree vertices. It is evident that this cost
is definitely less than Opt due to the metric property.
Therefore $Opt_{odd} \le Opt$.\\
The number of edges in that path will be $2*k-1$.\\
The path is composed of a $k-1$ matching and a k matching.
Hence the cost of this path is atleast twice the mincost
matching with $k-1$ edges. This implies
($min_{k-1}$ denotes mincost $k-1$ matching)
$Opt_{odd} \ge 2*min_{k-1} \implies min_{k-1} \le Opt_{odd}/2 \le Opt/2$.\\
Let us find the mincost matching with $k-1$ edges in the following manner :\\
\begin{verbatim}
min = sum over all the edges
for each subset of 2k-2 of the 2k odd vertices :
    min = minimum(mincost maximum matching over these 2k-2 vertices, min)
\end{verbatim}
The running time of the above algorithm is (2k choose 2)*p(n)
where p(n) which is the running time of mincost maximum
matching. It is easy to note that the algorithm correctly
finds the minimum matching with $k-1$ edges and runs in
polynomial time. Now add the edges of this matching to G'.
Now we have a connected graph where each vertex has even
degree except two vertices. These two vertices are the ones
which are left over in the matching. Hence there exists an eulerian
trail covering all these vertices starting from one of the odd
degree vertices. The cost of this eulerian trail is cost(G') + $min_{k-1}$.
Now by repeatedly short circuiting in this trail we will be able
to get a hamiltonian path due to the metric property and we are still
ensured that the cost doesnt increase. Therefore cost
of the obtained path $is \le cost(G') + min_{k-1} \le Opt + Opt/2 = 3/2*Opt$.
Hence we obtained an answer atmost 3/2 times optimum.\\

Case 2 : When one end point is specified :\\
Let specified vertex be v. Now consider the minimum
spanning tree of the subgraph containing the other
$n-1$ vertices. Now add the least edge from v to this
spanning tree to obtain a tree G'. Consider the optimal
path which starts from v. The cost of this path is
definitely is atmost the cost of G'. Hence $Opt \le G'$.\\

Now there will be some even number of odd degree vertices in G'
which includes v, let them be 2k. Let optimum hamiltonian
path across these vertices starting from v be $Opt'$.
Clearly $Opt' \le Opt$.
The optimal among these vertices consists of a k matching
which contains v
and a $k-1$ matching not containing v
. Now find mincost maximum matching
of the other $2k-1$ vertices. This will give us $k-1$ edges
which cover the $2k-2$ vertices. The cost of these edges
is lesser than both the k matching and the $k-1$ matching
described above. Therefore $2*min_{k-1} \le Opt' \le Opt$.
Now add these edges to G' and then we get a graph with all vertices
having even degree except v and another vertex. Now start from
v and complete the tour by short circuiting the eulerian trail
we get. The cost of this $is \le cost(G') + min_{k-1} \le Opt + Opt/2$.
The cost of this tour is less than $3/2*Opt$. Hence a 3/2 factor is obtained.
\part
Let s and t be the terminal vertices given.
Find the minimum spanning tree of the graph G say G'.
Clearly $Opt \le c(G')$.\\
Now consider all the odd degree vertices and even degree end points(either s or t)
in a set S. Now find the mincost maximum
matching of S. Let this matching be G''. Now if we add G' and G''
all vertices will have degree 2 except the two endpoints. Now
we can have an eulerian trail starting from s and ending at t. If we repeatedly
short circuit we would end up having a hamiltonian path with atmost
cost of c(G') + c(G'').\\
Assume that the optimum path is H.\\
As we have seen before $c(G'') \le TSP_{opt}/2$.
Now H and the edge $s-t$ form a cycle with cost atmost $TSP_{opt}$.\\
Therefore $c(G'') \le (c(H) + c(s-t))/2 \implies 2*c(G'') \le c(H) + c(s-t)$.\\

Property : The mincost maximum matching of $any$ subset of vertices
of any graph Q will have a cost of atmost half that of the $TSP_{opt}$.
Now we already know by the double touring mechanism that
$TSP_{opt} \le 2*c(MST)$. Therefore mincost maximum matching of any subset
of the graph Q is less than the cost of MST given metric constraints.\\

Consider $G''' = G' - (s-t)path$ in G'. Degree of some vertices
other than the endpoints have been reduced by 2. This implies
parity of none of the vertices change except that of the endpoints.
The parity of both the endpoints change. Hence it is easy to note
that the odd degree vertices here is exactly same as S.
Consider any forest T in G'''. The mincost maximum
matching of odd degree vertices in T is less than the cost of T due
to the property we see above. Hence consider the union of all mincost
maximum matchings of all forests in G''' which would be
less than c(G'''). Clearly
the sum over all these costs is less than the mincost maximum
matching of S. Therefore $c(G'') \le c(G''')$.\\
$c(G''') = c(G') - c(s-t)$ path. But $c(H) \ge c(G')$
and c($s-t$) path $is \le c(s-t)$ edge because of the metric
property.\\
Therefore $c(G''') \le c(H) - c(s-t) \implies c(G'') \le c(H) - c(s-t)$.\\

Therefore $2*c(G'') + c(G') \le c(H) + c(s-t) + c(H) - c(s-t) \implies
3*c(G'') \le 2*Opt$.\\
Hence $c(G') + c(G'') \le Opt + 2*Opt/3 = 5/3*Opt$.\\
Therefore 5/3 factor is proved.
\end{parts}

\question
\begin{parts}
\part
Assuming that V = {1...n} and V is a DAG,
construct the graph G' =(V',E'), where
V' = ${x_0...x_n} \cup {y_0...y_n}$,\\
E' = $\{(x_0, x_i) : i \in V \} \cup \{(y_i,y_0) : i \in V\} \cup \{
(x_i, y_j) : (i,j) \in E\}$.\\
Let all these edges have a weight 1.
Now let us prove that the max flow from $x_0$ to $y_0$ is k iff the minimum
path cover is $n-k$.\\
Consider a valid flow from $x_0$ to $y_0$. Now include
all the edges chosen in the matching. That is if $x_i$ and $y_j$ are
connected, choose edge $i-j$. It is given that the given graph is a DAG,
hence it cannot have any directed cycles. If a directed non
cycle has to be an undirected cycle then it implies atleast one of the
vertices should have two edges coming from it. But this is not
possible in the set of edges which were chosen by us as it is a
matching. The chosen edges cannot form a tree $non-path$ because
in a tree, two edges cannot emanate from the same vertex. Hence the
edges which we choose definitely form a disjoint
set of paths. Let us assume that t paths are formed
from the set of k edges. Each path covers one more
vertex than the number of edges in that path. Therefore
the number of vertices that are covered by the t vertex disjoined paths are
$n-k+t$(it is easy to see that this is not
more than n). Now $t-k$ vertices are uncovered. Cover all of them with path
length 0 paths. Hence we finally have $n-t+t-k = n-k$ number of paths covering
all the vertices. Therefore given any flow of k from $x_0$ to $y_0$ then
there exists a path cover of size $n-k$.\\

Consider the minimum path cover of all the vertices of size k.
In any of the paths the number of vertices covered is 1 more
than the number of edges in that path. The total number
of vertices covered is n then the number of edges chosen are $n-k$.
Now add all the edges in the cover in the bipartite graph created above.
As in disjoint path covers, no vertex is a source twice or sink
twice, the above forms a valid flow with value $n-k$. Therefore
the minimum path cover has a flow layout corresponding to it and
every flow corresponds to one path cover. As the flow
increases the path cover size decreases and hence the max flow
corresponds to the minimum path cover.\\

\part
Consider the trivial graph where $0 \leftarrow 1$
and $1 \leftarrow 0$. Now by the graph we made above, the
max flow would turn out to be 2 and the answer we get is 0. But the minimum
path cover is 1. Hence the above algorithm works correctly only for DAGs.
\end{parts}
\question
\begin{parts}
\part
Set cover doesnot have c ln n approximation for all constant c
unless P = NP.\\
Given any weighted set cover problem in the following
manner :\\
S = universal set of all elements to be covered($e_i$).\\
$S_i$ = some subset of S.\\
$W_i$ = weight of $S_i$.\\
Transform it into a $node-weighted$ steiner
tree problem as follows :
\begin{itemize}
\item Let all the $S_i$ denote vertices $V_i$.
\item Let all the $e_i$ denote vertices $v_i$.
\item Now $v_i$ is connected to $V_i$ iff $e_i \in S_j$.
All $V_i$ are connected to each other.
\item All the edges have weight zero.
Weights of all the $V_i$ are the weights of the corresponding $S_i$
and the weights of each of the $v_i$ is set as 0.
\item $v_i$ denotes the set of required vertices in the steiner
tree.
\end{itemize}
Note : Given any subset of $V_i$ which cover
all $v_i$, we can easily get a spanning tree out of it.\\

Consider any solution of the above steiner tree problem.\\
The solution would contain a set of $V_i$ and all $v_i$.
As all the edges have zero weight the cost
of the edges contribute zero and hence the cost
of the steiner tree is exactly the cost of $V_i$ chosen.
Now choose all the sets $S_i$ which correspond to
$V_i$ chosen above. Because the tree chosen is a
valid steiner tree it has to have edges to all the $v_i$
which implies $S_i$ got covers all $e_i$ and the $S_i$ has the
same cost.\\
Similarly we can show that for every solution of
the Set cover there is an equivalent steiner tree solution
(Pick all $V_i$ corresponding to $S_i$ chosen). Hence
the above is a valid reduction from Set cover to $node-weighted$
steiner tree and preserves the cost. Therefore if
$node-weighted$ steiner tree has a c ln n approximation
for all constants c, then Set cover has a c ln n
approximation for all c which would imply P = NP.\\
Therefore there exists c such that a $c$ ln n
approximation doesnt exist for $node-weighted$ steiner
tree problem unless $P = NP$.
\part
\end{parts}
\end{questions}
\end{document}
