\documentclass[12pt]{exam}
\printanswers
\usepackage{amsmath,amssymb,complexity}
\usepackage[T1]{fontenc}
\begin{document}

\hrule
\vspace{3mm}
\noindent
{\sf IITM-CS6840-2015 : Advanced Complexity Theory  \hfill Given on: Feb 15, 10:30am}
\vspace{3mm} \\
\noindent
{\sf Problem Set \#2 \hfill Due on : Feb 22, 10:00pm}
\vspace{3mm}
\hrule

\begin{questions}

%\question[5]
%Show that computing the permanent of matrices with integer entries is in %$\FP^{\#\SAT}$.

%\begin{solution}
%Write Solutions like this.
%\end{solution}

\question[10]
Consider the following definition of $\IP^*$, a perfect soundness version of $IP$ : $L \in \IP^*$ if there exists a $(P,V)$ with $V$ being a probablistic Turing machine running in polynomial time in $n = |x|$.
\[ x \in L \implies \Pr_y~[(P-V_y) \textrm{ Accepts }] \ge \frac{3}{4} \]
\[ x \not\in L \implies \Pr_y~[(P-V_y) \textrm{ Accepts }] = 0 \]
Prove that $\NP = \IP^*$.
\begin{solution}
Let us first prove NP is contained in IP*.\\
Given NP machine M, forget the randomization and simply
ask for which path to run on from the prover and run on that
particular path and accept if accept reject if rejects. Now if there
exists an accepting path then prover will send it and if none are present
it cannot send any valid path and the verifier will find out, hence with
perfect soundness and perfect completeness without randomness, we can
realize NP. Hence $NP \subseteq IP*$.\\

Now let us prove that IP* is contained in NP.\\
Consider any IP* machine M, we will make NP machine
M' recognizing the same. Whenever M expects a proof, guess
the proof using the nondeterminism to run over all proofs and whenever the
verifier uses his randomness, again use nondeterminism to simulate parallel
running of all randomness and in each path accept if it leads to an accept
and reject otherwise. If M' doesnot accept x on any path
then M couldnot have accepted x as for all x there doesnot exist
a proof. If M' accepts x and M does not accept x it means there exists a
conversation between prover and verifier which could have possibly lead
to an accept but this is a contradiction to perfect soundness. Hence if
M' accepts x then M doesnot accept x. Hence M accepts x iff
M' accepts x implying L(M') = L(M). Hence $IP* \subseteq NP$.\\

Therefore NP = IP*.
\end{solution}
\question[10]
Prove that if $\NP = \RP$, then $\AM = \BPP$.
\begin{solution}
If $NP = RP$ then $NP \subseteq BPP$.
But we know that $BPP^{BPP} = BPP$. This means
$NP^{NP} = \Sigma_2 \subseteq BPP^{BPP} = BPP$.\\
We further know that BPP is closed under complimentation,
hence $\Pi_2 \subseteq BPP$. Therefore assuming
$AM \subseteq \Pi_2$ from question 3 we get,
$AM \subseteq \Pi_2 \subseteq BPP$. Hence $AM \subseteq BPP$.
We already know that $BPP \subseteq AM$ (discussed in class).
Therefore AM = BPP.
\end{solution}
\question[10]
Show that $\AM \subseteq \Pi_2$ and $\AM \subseteq \NP/\poly$.
\begin{solution}
AM is defined as follows :
\begin{itemize}
 \item If $x \in L$ then for all r there exists a $\pi$ such that V(x,r,$\pi$)=1.
 \item If $x \not \in L$ then $Pr[\exists \pi : V(x,r,\pi) = 1] \le 1/3$.
 \end{itemize}
Let us amplify the soundness to suit as follows :\\
$x \not \in L \implies \forall \pi P(\exists V(x, r, \pi) = 1) \le 1/(2^{p+1}).$\\
$R_\pi = \{r | \forall V(x, r, \pi) = 1\}.$\\
Now we can actually send the random bit before hand even because the
number of wrong answers(total) would be $R*2^p/2^{p+1}$ which is R/2.\\
This proves that more than half of the random strings donot get accepted
by any proof(from $AM_\epsilon = AM$ proof). Hence the quantifier
statement $\forall r \exists \pi V(x, t, \pi) accepts$ is not true.
When x is in the language then as per definition,
$\forall r \exists \pi V(x, t, \pi) accepts$ (Q). Hence given any AM protocol,
the quantifier statement Q realizes it and Q is in $\Pi_2$. Hence
$AM \subseteq \Pi_2$.\\

Equivaletly we can see AM as given x it asks for an advice
(r) such that if NDTM M which runs over all proofs and verifies
V(x, y, r). If V(x, y, r) satisfies it accepts else it rejects.
Proof for the above :\\
If x is in the language no matter what r is given it will
lead to an accept so anyway M will accept. When x is not
in the language we know that there exists atleast one r for which
V(x, y, r) rejects and each time the advice would be this r.\\


Let N be a matrix of size $2^n*2^{p(n)}$.\\
and probability error = $1/2^{2n}$.
Here the number of rows is $2^n$ and number of columns is $2^{p(n)}$.\\
N(x,y) = 1 if random string y leads M to correct decision on input x
and 0 otherwise.
Let us look at how many zeroes are there in any row. There are $2^p(n)$ total strings possible and based on our assumption there can be atmost $2^{p(n)}/2^{2n}$.
Hence summing over all rows we end up with $2^{p(n) - n}$ maximum number of zeroes
where we can easily see that it is less than the number of columns. Hence there is
atleast one column full of ones which essentially gives us a set of strings which
helps us find the correct path $always$ because of the construction of the matrix.

(the above argument is same as showing BPP is in P/Poly).
Hence we have shown that there exists r for every |x| such that NP machine M
is lead to accept on r. Hence $AM \in NP/Poly$.


\end{solution}
\question[10]
Show that for all constant $k \ge 2$, $\AM[k] \subseteq \AM[\frac{k}{2}+1]$. (Hint(*) : Divide the protocol sequence into $\frac{k}{4}$ AMAM segments. Replace each MAM with AMA. This does not work as it is.)

%\question[10]
%Let $S$ be a set of $n$ elements (numbered $1$ to $n$). Let $G$ and $H$ be groups of permutations on $n$ elements. That is $G$ and $H$ are subgroups of $S_n$. The right coset $G$ with respect to of $g \in S_n$ is defined as $Gg = \{ g'g \mid g' \in G \}$. The coset intersection problem asks, given $G, H, g, h$, does $Gg$ intersect $Hh$? It is known that Graph Isomorphism problem reduces to Coset intersection problem.
\question[10]
Show that if $\PSPACE \subseteq \P/\poly$ then $\PSPACE = \MA$.
\begin{solution}
Consider $L \in PSPACE$. As IP = PSPACE, we can have an interactive
proof system for L where the prover runs a PSPACE machine. If L is in P/Poly,
then there exists a polynomial size advice for every input length which
can be used to verify whether $x \in L$ or not. Consider the IP prover
verifier system for L say M. We know that the prover doesnot do anymore than
what a PSPACE machine could. If we could transfer the power of this
PSPACE machine to the verifier somehow then the verifier will not ask
the query to the prover but simulate it on the verifier side itself.\\

Let Merlin give the advice for |x| = p(n)*q(n) for the PSPACE complete machine
where p(n) denotes the number of times we query the prover in IP system
and q(n) denotes length of the conversation. Hence any question asked to
the prover cannot be more than this. If the prover sends this, then
the verifier will just start simulating the IP machine for L all
by itself and hence making it an MA protocol. If x is in L
then and the proof being the above specified
advice, then perfect completeness is guaranteed as IP macine has perfect
completeness. If x is not in L due to the IP system which we establish,
the prover cannot cheat us more than a specified less probability (like 1/3).
Hence the above MA protocol recognizes any L in PSPACE.
We already know that $MA \subseteq IP = PSPACE$. Hence
MA = PSPACE.
\end{solution}
\end{questions}
\end{document}
