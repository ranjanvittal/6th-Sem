\documentclass[solution,addpoints,12pt]{exam}
\printanswers
\usepackage{amsmath,amssymb}
\usepackage[T1]{fontenc}
\begin{document}
\hrule
\vspace{3mm}
\noindent
{\sf IITM-CS6840-2015 : Advanced Algorithms  \hfill Given on: April 9}
\vspace{3mm} \\
\noindent
{\sf Problem Set \#1 \hfill Due on : April 29}
\vspace{3mm}
\hrule

\begin{questions}
\question
\begin{parts}
\part
1/3 CNF SAT = 3 CNF SAT with the additional constraint that every clause
has exactly one literal satisfied.\\

1/3 T CNF SAT is one in which all the literals are forced to be true.

Consider the following reduction from 1/3 CNF SAT to 1/3 T CNF SAT as follows :\\

Let input variables be $x_i$'s. Now in the formula of 1/3 CNF
replace every $\overline{x_i}$ by a new variable $y_i$. Now add clauses
as follows :\\
($x_i$ + $y_i$ + f) for every variable $x_i$
and finally a clause (f + f + t).\\
It is easy to see that the now modified into an instance which is 1/3 T CNF SAT
as all literals are positive.

If 1/3 CNF says yes, then there exists a satisfying assignment such that
exactly one of the literals in each clause are true. Therefore substitute the
values according to the satisfying assignment by maintaining $y_i = \overline{x_i}$.
Substitute f as false and t as true. As $y_i = \overline{x_i}$, all the
extra clauses added of the form ($x_i$ + $y_i$ + f) will be satisfied (according to 1/3 CNF)
and (f + f + t) will also be satisfied.
Therefore if 1/3 CNF says yes then 1/3 T CNF for the modified formula also says yes.\\

If 1/3 T CNF says yes :\\
This definitely means that f has to be assigned to false and
t has to be assigned to true as f + f + t was satisfed. Further this
enforces $y_i$ = $\overline{x_i}$ as exactly one of them has to be true
as ($x_i$ + $y_i$ + f) is satisfed and f is false. Therefore given a setting
to these variables we can map all the $x_i$'s of the current satisfying setting
as the variables to 1/3 CNF SAT and it would satisfy all the clauses as 1/3
T CNF satisfies the other clauses as well and $y_i = \overline{x_i}$ is ensured.

Therefore 1/3 T CNF is NP hard as 1/3 CNF is NP complete (easy to see that the reduction
is in polynomial time).\\
It is easy to see that checking whether an assignment is correct for
1/3 T CNF is in P. Therefore 1/3 T CNF is NP complete.
\part
This problem is a generalized version of the first part (like 1/k T CNF). That is consider
the following setup :\\
For every person assign a variable and every group is represented as a clause
as 'or' of the person's variables in the group (disjunction). Now take
an 'and' over all such clauses and we would get a formula which is satisfiable
iff every clause has exactly one true literal which essentially implies that
each circle has exactly one guy chosen from it. Therefore 1/3 T CNF SAT
is reducible to 1/k T CNF SAT as 1/3 T CNF SAT is a specific such instance.
Therefore 1/k T CNF SAT is NP hard but hecking whether a given assignment
is satisfiable is in P and hence 1/k T CNF SAT is in NP. Therefore 1/k
CNF SAT is NP complete. Therefore there wouldnt exist an efficient algorithm for
solving this unless P = NP.
\end{parts}
\question
\begin{parts}
\part
Consider the following reduction from 3 CNF SAT to majority SAT as follows :\\
To every clause $C_i$ which is like say ($z_i$ + $z_j$ + $z_k$) where
$z_i$ represents a variable or its negation,
add two more variables $y_1$ + $y_2$ to the clause making it ($z_i$ + $z_j$ + $z_k$ + $y_1$ + $y_2$)
and feed it into the majority SAT solver.\\

If 3 CNF SAT has a satisfying instance, assign $y_1$ = 1 and $y_2$ = 1 and substitute
the satisfying assignment got from 3 CNF SAT in the formula we get after applying the
above modification. Now every clause in the 3 CNF instance has atleast one positive
literal and we have enforced two more positive literals in the formula made up
for Majority SAT ($y_1$, $y_2$). Therefore every clause has atleast 3 true literals
and hence is a satisfying instance of majority SAT.\\

If majority SAT has a satisfying instance, then every clause has atleast 3 true literals.
This means that apart from $y_1$ and $y_2$ some other literal in the clause also
must evaluate to true in every clause. Therefore the original 3 SAT instance will
also be satisfied with the same setting of variables which we get from majority SAT
if majority SAT returns true. Therefore 3 SAT would also say true for its given input.\\

It is easy to see that the reduction is in polynomial time and as 3 CNF SAT
is NP complete, Majority SAT is NP hard. Given am assignment
checking whether majority SAT is satisfied by the assignment is trivially in P.
Therefore majority SAT is in NP. Therefore majority SAT is NP complete.
Therefore unless P = NP, we cannot expect to find a polynomial time
algorithm for this. We can get an exponential time algorithm though ($2^n*p(n)$ where
n is the number of literals and p(n) is the
time required to check each clause ) for solving this problem.

\part
We can propose an algorithm which can give us a half approximation as follows :\\
Every clause can be majorly satisfied with a probability of 1/2. This is because
either 0 is major occuring element or 1 is the major occuring element in each clause.\\
Let the given formula have m clauses and n variables.
Therefore number of assignments which are satisfiable for each clause is (1/2)*$2^n$ = $2^{n-1}$.\\
Therefore sum total of all the clauses satisfied over all the assignments is
exactly $2^{n-1}$*m. Total average number of clauses satisfied per assignment
= $2^{n-1}*m/2^{n}$ as $2^n$ is the overall number of assignments possible. Therefore
average number of clauses satisfied = m/2. Therefore a randomized algorithm
can give us an assignment in expected polynomial time with m/2 clauses satisfied
and hence gives us a half approximation (as maximum clauses which could be satisfied
is m).\\
The above procedure can be derandomized by the method of conditional probability
(ie) at each step you fix one variable and move in the direction in which
the expected number of clauses satisfied is higher. Therefore this algorithm
is a polynomial time 1/2 approximation algorithm by the method
of conditional probability.
\end{parts}
\end{questions}
\end{document}
