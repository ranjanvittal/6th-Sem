\documentclass[12pt]{exam}
\printanswers
\usepackage{amsmath,amssymb,complexity}

\usepackage{hyperref}
\hypersetup{
    colorlinks=true,
    linkcolor=blue,
    filecolor=magenta,
    urlcolor=cyan,
}
\begin{document}

\hrule
\vspace{3mm}
\noindent
{\sf IITM-CS6840-2015 : Advanced Complexity Theory  \hfill Given on: Mar 23, 11:55pm}
\vspace{3mm} \\
\noindent
{\sf Problem Set \#4 \hfill Due on : Apr 1, 11:55pm}
\vspace{3mm}
\hrule

\begin{questions}

\question[10]
In this problem, we will show that there is a circuit of
depth $d$ and size $\poly(n).2^{n^{\frac{1}{d-1}}}$ which computes
parity on $n$ inputs. Assume that the circuit uses variables $x_1, x_2
\ldots, x_n$ and $\lnot x_1,\lnot x_2, \ldots, \lnot x_n$
\begin{parts}
\part[2] Show that PARITY can be computed by a circuit of depth 2 and
size $c.2^n$ where $c$ is a constant, thus proving the result for
$d=2$.
\part[3] Show how to reduce the size to $2^{\sqrt{n}}.\poly(n)$
  increasing the depth to 4.
\part[2] Show how to do the same with depth 3 thus proving the result for $d=3$.
\part[3] Give a generalization to other depths.
\end{parts}
\begin{solution}
\begin{parts}
\part  We can write the disjunctive normal form of parity that
is sum of products form of parity. This would require at the most
$2^n$ or gates with fan in of n and 1 or gate with fan-in of
atmost $2^n$. Therefore total number of gates required is $2^n + 1$
upper bounded by 2*$2^n$ gates anyway.
\part
Consider the following idea :\\
Split n as $n^{0.5}$ groups each containing $n^{0.5}$ variables each.\\
Now for each of the groups use $c*(2^{n^{0.5}})$ gates as done above
using DNF to compute the xor of the variables in each group.
This requires a depth of 2 as seen above and requires $c*n^{0.5}*2^{n^{0.5}}$
number of gates.
Now we will have $n^{0.5}$ variable whose xor is to be computed.
Again paste the circuit which uses DNF to compute the xor of these
$n^{0.5}$ bits using $c*(2^{n^{0.5}})$ gates and a further
depth of 2. Hence we will finally get the xor output of all the
bits at a depth of 4 and number of gates utilized would be
$c*2^{n^{0.5}}*(n^{0.5}+1)$.\\
Hence number of gates used is $2^{n^{0.5}}*poly(n)$ with a depth of 4.
\part
Now a simple trick can transform this into a 3 depth circuit.
Instead of doing the DNF again the second time, do a CNF of xor.
Now the different levels are : and, or, or, and (ie) the
first and, or is for generating the xor in the DNF form as mentioned
above and the second or, and is for generating xor in CNF for the
$n^{0.5}$ bits gained by xor in the first step. Now we can
easily see that the two consecutive or's can be merged into 1
as instead of taking two or's we can simply increase the fanin and
take one or in that place. It is trivial to see that this decreases
the number of gates and hence is still $c*poly(n)*2^{n^{0.5}}$.
\part
Split n into $n^{(d-2)/(d-1)}$ groups of $n^{1/(d-1)}$ variables each.
Now each of these groups can be computed using a 2 depth circuit in DNF
and $c*2^{1/(d-1)}$ number of gates. Therefore with $c*n^{(d-2)/(d-1)}*2^{1/(d-1)}$
number of gates we can take the xor of these groups independently. Now
recursively reduce $n^{(d-2)/(d-1)}$ to $n^{(d-3)/(d-1)}$ in a similar
manner this will again be $poly(n)*2^{n^{1/(d-1)}}$ with an additional
increase of 2 in the depth. If we just continue so on, like
reducing $n^{(d-k)/(d-1)}$ to $n^{(d-k-1)/(d-1)}$ in every step from
k = 1, we would finally have a circuit which has $poly(n)*2^{n^{1/(d-1)}}$
number of gates and depth would be $2*d - 2$ (2 for every drop from k to k+1
(ie) from $n^{(d-k)/(d-1)}$ to $n^{(d-k-1)/(d-1)}$). Now by the
trick mentioned above, use the conjunctive normal form every alternative step.
This would transform :\\
$and_1$, $or_1$, $and_2$, $or_2$, .... and, or to\\
$and_1$, $or_1$, $or_2$, $and_2$, $and_3$, $or_3$, $or_4$, $and_5$, .... .\\
It is easy to note that every subsequent pairs of ors or pairs of and can
be combined like $or_1$ and $or_2$ can be combined and so on. It is
easy to note that there would be $d-2$ such subsequent pairs. Hence it
is easy to see that while merging these subsequent pairs, we would
end up reducing the depth to d while the size only goes down and
hence the size is still $poly(n)*2^{n^{1/(d-1)}}$.\\


\end{parts}
\end{solution}
\question[10]
Show that every sparse set can be computed by a polynomial sized
circuit family. (without using the result that $\P/\poly = $ {\sf PSIZE}).
\begin{solution}
A sparse set is one in which number of strings of length n which
belongs to L are polynomially many for all n say p(n). Hence consider
the infinite union of the following circuits all of which have polynomially
many gates :\\
For any n lets say p(n) n length strings are accepted.
Checking equality of two n bit strings
can trivially be done in polynomial time (check equality
of each bit by making the xnor gate and take an and over all such).
Now check equality of the input sequence of bits with each
of the p(n) length strings and have an or gate which feeds in
from all of these equality checks. Now this circuit exactly
accepts all the sparse set strings of length n and has only
polynomially many number of gates. For any n we can see that
polynomially many number of gates are sufficient to accept
all n length strings which belong to the sparse set.\\

Hence every sparse set can be computed by a polynomial sized
circuit family.
\end{solution}
\question[10]
A circuit is said to be {\em monotone} if it does not use negation
gates (all gates either $\land$ or $\lor$ type) and is said to be {\em
  weakly monotone} if all the negation gates are at the input.
\begin{parts}
\part[5] Argue that if a function $f:\{0,1\}^n \to \{0,1\}$ is computed
  by a circuit $C$ of size $s$ and depth $d$, then $f$ can also be
  computed by a weakly montone circuit of size $2.size(C)$ and depth
  $d$. Use this to show that circuit value problem even when restricted to weakly monotone
  circuits is $\P$-complete.
\part[5] Argue that the functions computed by monotone circuits are monotone (use the definition of monotone functions that we used in class). Hence conclude that parity cannot be computed by such
  circuits at all (given any size).
\end{parts}
\begin{solution}
\begin{parts}
\part
Given a circuit C, there exists a gate all of whose inputs are either
the input variables or its negations. If that gate is an and,
then make an or gate which looks at the negations of all the
input variables looked at by the and gate (similarly
if it was or, make an and gate which feeds in from all the
negations of the variables looked at by the or gate).It is easy by de morgans
law to notice that this or gate computes the negation of the previous
and gate. Now wherever this and gate was originally negated, feed in with
this or gate. Now this circuit(C') computes the exact same function as what the
original circuit computes. Now remove the gate we saw and its compliment from
C' and let all the fanout of these two gates in C' be treated as input
vatiables. Let this be C''. If C'' is computed with only and/or
gates with all the negations pushed to the input variables (say C'''), then that
circuit can be used in place of C'' in C' and we will get a circuit
which is weakly monotone (C''''). Another thing is that C' has the
same depth as C.\\
Now this circuit again has a gate all of whose inputs are either
input variables or its negations and we can again do the same
procedure mentioned above. Like this, each time find the gate, do the
above procedure and then remove the gate to get another circuit. Each
step we can notice that the correctness of the circuit is maintained
and that for every gate we are issuing exactly one more copy for the gate.
As in each step we are removing a gate, we would finally end up with
no gates finally at some stage. Now we can start reassembling the
circuit with every gate and its negation as seen above while maintaining
the correctness of the circuit. We would finally have a circuit which
has exactly twice the number of gates(and/or) in C as for every gate
we have introduced exactly one additional gate and the depth
also doesnot change. It is easy to see that given
a circuit computing a boolean function, making it
weakly monotone as seen above can be done in Log space. Therefore
as circuit value problem is P complete, weakly monotone
circuit value problem is also P complete.
\part
For n = 1, the function computed by a monotone
gate is monotone (n is the number of gates and fanin of this gate doesnot matter).\\
Let us assume for every k till n number of gates that if all the given
gates are monotone, the circuit can compute only monotone gates.\\

Consider a circuit C with n+1 monotone gates computing function f.
There exists a gate which
looks at all the input variables say gate g. Now let $y_1, y_2, ... y_l$
be the input variables to this monotone gate and let these sequence of
bits be denoted by y. Now remove
this gate from C and replace
all the fanout of this gate as input wires to obtain C'.
It is known that C' computes only a monotone function say f'
by induction assumption.
If $x' \le z'$ then $f'(x') \le f'(z')$ where x' is an input to
circuit C'.\\
Consider $x \le z$ as input to f. This means that $y(x) \le y(z)$
where y(x) denotes the $y_1, y_2, ... y_l$ of x and similarly
for y(z).\\

Case 0 :\\
x = z is trivially done as f(x) = f(z).\\

Case 1 :\\
y(x) = y(z) and $x < z$.\\
In this case the output of the gate for both x and z will be the same.
(ie) g(x) = 0 iff g(z) = 0 where g is the gate looked at.
Hence if we remove this gate to obtain C', all the fanouts
of this gate will be either uniformly zero or uniformly 1.
Let the input to the C' be x' from x and z' from z.
It is easy to see that $f'(x') = f(x)$ and $f'(z') = f(z)$.
Now in the transformation from C to C', we might have lost some
input variables(I) which fed into g and would have gained new
variables which are the fanout of this gate (O).\\
All the variables except I and O are retained. We already know
that I(x) = I(z) and O(x) = O(z) as y(x) = y(z). Therefore
$x' < z'$. As $x' \le z'$, $f'(x') \le f'(z')$ as f' is monotone.
Therefore $f(x) \le f(z)$.\\

Case 2 :\\
$y(x) < y(z)$.
This implies $g(x) \le g(z)$. Hence if we remove this
gate to obtain C', all the fanouts of z would be greater
than or equal to all the fanouts of x. Now in the transformation from C to C', we
might have lost some input variables(I) which fed into g and would have gained new
variables which are the fanout of this gate (O). Now all the other
variables are retained(R). $O(x) \le O(z)$ as $g(x) \le g(z)$. Therefore
as $x \le z$, $R(x) \le R(z)$, and $O(x) \le O(z)$ implies $x'\le z'$
where x' and z' can be seen as input to C'. Therefore $f'(x') \le f'(z')$
and $f(x) \le f(z)$ follows similarly to case 1 as $f'(x') = f(x)$
and $f'(z') = f(z)$.\\

Therefore if $x \le z$ then $f(x) \le f(z)$. Hence f is monotone.
Therefore if all boolean circuits with
k monotone gates are monotone, then all boolean circuits with
k+1 monotone gates are monotone and all boolean circuits with
one monotone gate is monotone. Therefore any boolean circuit
with any number of monotone gates is monotone.
\end{parts}
\end{solution}
\question[10]
It is unknown currently whether there is an explicit function in $\NP$ that does not have $O(n)$ sized circuits. In this problem, we will show something weaker. Show that there is a function in the $\PH$ which does not have
$O(n)$ sized circuits. (Hint: how does one describe a Boolean function which can be computed by a circuit of size $k$, but cannot be computed by any circuit of size $k'$ where $k' < k$? think of $k$ as $O(n^2)$ and $k'$ as $O(n)$.)

\question[10]
Read up on {\em Characterisation of regular languages using syntactic monoids.}. Link : \href{http://www.cmi.ac.in/~kumar/words/lecture01.pdf}{www.cmi.ac.in/\~{ }kumar/words/lecture01.pdf}\\
Use this to show that all regular languages are in $\NC^1$.
\begin{solution}
We know that ${NC}^1 = BF$. Therefore if we are able to
make a log space constructible BF, we are done.\\
Wlog the input is binary.\\

Let the regular language be accepted by an automaton with k states
namely $x_1, x_2, ... x_k$.\\
Consider the following k+1 to k bit function.\\
k wires ($iw_1, {iw}_2, ... {iw}_k$) denote the k states of the machine such
that if ${iw}_i = 1$ implies that the current state is $x_i$. It is guaranteed
that exactly one of these wires in the input are 1. The other input is from the
input to the automaton (ie) it looks at one bit from the input say $\alpha$
. The outputs (${ow}_1, {ow}_2 ... {ow}_k$) are again
such that exactly one of these wires are 1. If ${ow}_j$ is 1, it would denote the state
to which the automaton jumps to from state $x_i$ is $x_j$.\\
This can be achieved as follows :\\
Let $x_{i_1}, x_{i_2}, .... x_{i_l}$ on 0 goes to $x_t$ and
$x_{j_1}, x_{j_2}, ..... x_{j_q}$ on 1 goes to $x_t$.\\
Then ${ow}_t = (x_{i_1} + x_{i_2} + .... x_{i_l}).\overline{\alpha} + (x_{j_1} + x_{j_2}
+ ..... x_{j_q}).\alpha$ (where + denotes or and . denotes and) . It is
easy to see that all transitions of the automaton can be taken care of by
this boolean circuit. Also only one output wire would be one because
only one input wire was one and automaton on an input bit goes to only
one other state. This circuit requires constant number of gates as k is
a constant.\\

Hence consider the following construction of a circuit which has fanout = 1
for all gates and has polynomially many gates.\\
There are n such circuits where n denotes the input size length. The
first gate looks at the first bit of the number and the rest
all input wires are pre encoded to zero except the input wire
which corresponds to the start state which is assiged to 1.
For every other $i^{th}$ circuit the input is the output of
the ${i-1}^{th}$ gate and the $i^{th}$ bit of the input. After
the $n^{th}$ circuit 'or' all the wires which corresponds to final states.\\
The output wire, which is 1, of the $i^{th}$ circuit would tell us which state the
automaton is in after seeing the $i^{th}$ input bit. Hence if
the string belongs to the language, it would end up in a final state and hence
our circuit would output 1 indicating acceptance and if a string
doesnot belong to the language, it would not end up in a final state
and our circuit would output zero indicating rejection.\\

This circuit has O(n) number of gates and no gate has fanout greater than 1.
It is also easy to see that this circuit is log time constructible
(actually log time constructible).
Therefore this circuit is in BF. Therefore there exists a $\NC^1$
circuit which computes the same. Hence all regular languages are in
${NC}^1$.


\end{solution}



\end{questions}

\end{document}
