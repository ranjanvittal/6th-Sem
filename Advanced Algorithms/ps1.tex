\documentclass[12pt]{exam}
\printanswers
\usepackage{amsmath,amssymb,complexity}
\begin{document}

\hrule
\vspace{3mm}
\noindent
{\sf IITM-CS6840-2015 : Advanced Complexity Theory  \hfill Given on: Jan 27, 07:30am}
\vspace{3mm} \\
\noindent
{\sf Problem Set \#1 \hfill Due on : Feb 04, 07:00am}
\vspace{3mm}
\hrule

\begin{parts}
\part
Given a graph G, do BFS starting from any node. Let the root be coloured
black and all its children be coloured red. Basically any node in 2*i+1
level is coloured black and any node in 2*i+2 level is coloured red. Now while
doing bfs, if we encounter a node which initially occurred at an odd level
occurring at an even level or vice versa, this implies there exists an odd cycle
in the graph and the graph cannot be $2-coloured$. Otherwise if we go on
colouring in this manner we would end up colouring all the vertices either
red or black and no black and red vertices will have edges between them.
\part
The decision version is as follows :\\
Can you colour the graph using k colours with no two sharing an edge
having the sme colour ? is the decision version of the same problem.\\
Assuming the Graph colouring problem is polynomial time solvable using
Algorithm A:\\
Given any G, k forward G to A. It would gives us a number k' denoting the
minimum number of colours used to colour the graph. If $k < k'$ then
definitely G is not $k-colourable$ while if $ k \ge k'$ then the graph
can be coloured using k colours. Hence decision version is solvable
if Graph colouring is.\\
Assuming decision version is solvable using algorithm A in polynomial:\\
The graph can definitely be coloured with n colours
where n denotes the number of vertices that is every vertex
having a different colour. Now given graph G, ask decision
version (G, 0), (G, 1), (G, 2) so on till (G, n). Now till some
$k-1$ the algorithm A would say no and from k it would start saying yes.
This means that the graph was not $k-1$ colourable while it is indeed
$k-colourable$. Hence the minimum number of colours required to colour the
graph is k. The running time of this machine is polynomial composed
polynomial which is still polynomial hence the
Graph colouring problem is solvable in polynomial time if decision version is.\\
\part
It is given that $3-colouring$ problem is $NP-Complete$. Consider the
following reduction from $3-colouring$ to decision version :\\
Given Input G, give input as (G, 3) to the decision version
the decision version will say yes iff there exists a $3-colouring$ possible.
Hence $3-colouring \le_m decision$ version. Hence the decision version is $NP-Complete$.
\part
Consider a valid $3-colouring$ c of the graph G.
Assuming RED is coloured with red, if one of vertices $x_i$ or $\overline{x_i}$
are coloured red then the colouring rule is not satisfied as RED has an edge to
all these variables. Hence each other variable is either coloured either
true or false. Now both $x_i$ and $\overline{x_i}$ cannot be coloured
the same colour because they share an edge between them. Hence one of them
will be coloured true and the other false.\\
Given any satisfying assignment to the problem, colour $x_i$ to be false
if in the satisfying assignment it indeeds turns out to be so and colour
$\overline{x_i}$ to be true and do similar if $x_i$ is true. Repeat this
for all the vertices. Hence this will produce a valid 3 colouring
between containing only the literal edges as 3 colouring condition will
be satisfied within the literal edges.
\part
Let vertex connected to x be $v_1$, connected to y be $v_2$, connected to
$v_1$ and $v_2$ be $v_3$, connected to $v_3$ be $v_4$ and that connected
to TRUE and $v_4$ be $v_5$.\\
Assuming all x, y, z are coloured false, then
$v_5$ has to be coloured red which implies $v_4$ has to be coloured
false. Now none of $v_1$ or $v_2$ or $v_3$ can be coloured false because
one of each's neighbours are coloured false. As these 3 form
a triangle and one colour is not usable, this widget is not
3 colourable.\\
Assuming z is coloured true :\\
If c(x) != c(y) (c denotes colour), then colour $v_5$ red,
$v_4$ false and $v_3$ red. Now we know x and y can be either coloured true or false.
then colour $v_1$ as compliment of
x (ie) if x is true colour $v_1$ false and so on and colour $v_2$ as
compliment of y. Now this widget is completely coloured using 3 colours.\\
If c(x) = c(y), colour $v_2$ to be compliment of c(y), colour $v_1$ red,
colour $v_3$ same as y and colour $v_4$ red and $v_5$ false.
Hence in this case it is 3 colourable.
Assuming exactly one of x or y is coloured true and z is coloured false:\\
we can easily check that the following strategy works :
(r denotes red, f denotes false and t denotes true)
c($v_5$) = r\\
c($v_4$) = f\\
c($v_3$) = r\\
c($v_2$) = $\overline{c(y)}$\\
c($v_1$) = $\overline{c(x)}$\\
Assuming both are coloured true,\\
c($v_5$) = r\\
c($v_4$) = f\\
c($v_3$) = t\\
c($v_2$) = f\\
c($v_1$) = r\\
Hence the widget is $3-colorable$ if and only if at least
one of x,y, or z is coloured c(TRUE).
\part
Given that there exists a satisfying assignment accordingly colour
all the vertices as indicated in part d and use the schema in part e
to colour all the clause edges. Due to the valid consistency check we
are guaranteed to have a proper 3 colouring overall.\\
Given that there exists a valid $3-colouring$, we
already know that the literals will be coloured either
true or false and that the compliment colour is
also appropriate and that a clause widget
is 3 colourable iff atleast one of the literals in it is
true. Hence assign the variables as the colour suggests and
we are assured to end up with a valid satisfying assignment.\\
Hence by the above proof we have shown that $3-SAT \le_m 3-colouring$
and $3-SAT$ is $NP-Complete$. Hence $3-colouring$ is $NP-hard$.
There is a trivial $NP$ machine to do this problem as verifying
$3-colouring$ is in P. Hence $3-colouring$ is $NP-complete$.
\end{parts}
