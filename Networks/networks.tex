\documentclass[solution,addpoints,12pt]{exam}
\printanswers
\usepackage{amsmath,amssymb}
\usepackage[T1]{fontenc}
\begin{document}
Different layers are physical layer, link layer, network layer, transport layer,
session layer, presentation layer and application layer(Bottom to top).(OSI
layers)\\
TCP/IP stack was what which was implemented which has five layers. They donot
contain
presentation and session layer are not there. physical is L1, link is L2 and so
on.
We can have intermediate layers like L2.5 .\\
TCP is connection oriented and reliable and UDP is Connection less and
unreliable.
Native IP layer of TCP is connection less. But these days some connection
oriented to the
network layer are supported.\\

At application level we open a file read or write. A file is treated as a
sequence of
byte streams. A socket is a virtual interface which is a pipeline available to
communicate as read and
write. Assume there are 5 to 6 application write issued. In the network layer
we can send packets via
different paths (individual messages are sent). No notion of the order. TCP
maintains a buffer.
TCP has a sequence number for every message that is sent. So when it reaches a
stack of messages.
it orders it based on the sequence number. Hence connection orientedness is
maintained.
Hence application sees it in the correct order. UDP doesnot have the sequence
numbers and it
will just send them in the order it received which need not be the proper
order. Therefore
application layer has to sort it. TCP will request retransmission if a message
is not found. UDP wont request for retransmission.\\

There is some part of the kernel which runs the TCP process or IP process. IP
in routers
is quite complicated. FTP is a process. Some memory sharing system is required.
From application buffer
to kernel buffer it goes to the device buffer from which it is transmitted.
Packets, segments and frames are different names for messages.\\

Original packet is expanded. TCP has its own header(port number is added) and
many control checks are added to the messages.
UDP has only 8 bytes while TCP header can be 20 to 60 bytes. Client and server
talking to each other is called a session. IP level will have its own header (
source and destination IP address is added).\\
IPv4 header has 20 to 60 bytes and IPv6 header has 40 bytes is added to TCP.
Ethernet is limited to 1500 bytes. Hence we will use approximately 500 bytes.
Most of the control and fragmentation occurs at the IP layer. TCP or IP can
give 65536 - 60 - 60 bytes (removing headers) to the below layers.
The information above is not touched that is IP doesnot touch the packets sent
by TCP. IP only checks its
header alone and similarly every layer checks its header up.\\

Things present in the MAC layer are MAC address of source and destination. For
ethernet traditionally
mac header is 18 bytes. Mac address of source and destination is 12 bytes.
Anything with a connection(wifi cards etc) has a unique MAC address. We donot
require the MAC address for translation. For global
tranmission IP address is used and local MAC address is used to transmit within
a LAN. Destination MAC
address is that of the next hop of the packet (like 10.6.0.254). MAC address is
required only within the
local network. Adddress resolution protocol makes sure that the first
transmission of the MAC address.This is generally done on mass transmission.\\

If there are 5 routers inbetween source and destination. By the time a 4 byte
packet is sent
it will atleast be bloated to 64 bytes as it is the minimum requirement for
ethernet. If MAC address and destination field are properly filled it is sent
down to Link layer. Before this it also does error control and ensures
reliability. The link layer doesnot know where to send it next. In case the
packet is corrupted then there is a request to resend the message. Link layer
forwards it to the
physical layer which first checks all the addresses.\\

Ethernet comes into a router and gets routed to whichever IP it needs to be
forwarded next (can be another router). OSPF, RFP and all such protocols make
the forwarding table that is the one
which establishes connections between the input and the output of the router.
Network
layer should know the MAC address of the next hub. Generally all hubs have only
network, link
and physical layer. The original MAC address is now changes as the packet moves
on to the second router.
The header leaving from the original machine is ripped off and it is modified
to change the next MAC address. Based on the forwarding table the next target
router(IP) can be found out but the destination(final) MAC address is
maintained. Unique IP Addresses are assumed as input.\\

Network Address translation decides which all packets are to be sent through a
local area network.
The NAT has source IP, Destination IP, Source Port and Destination Port. IP
is not generally checked and it is also possible to push in any IP address you
want.\\

Finally it gets to the destination's transport layer. It will do all the checks
though before moving onto
the transport layer. Client has ftp, ssh and http servers and the server too.
At these places
we have multiplexers. We have ports as virtual transport level interfaces
sharing a given IP address.\\

TCP or UDP have 16 bit port numbers and and hence it can have 65536 different
port numbers. Every process
on a machine has port numbers associated with it. System processes run on port
numbers 0 to 1023. 1024 onwards user processes can use. 21 is ftp server, 22 is
ssh server, 80 is http
and 443 is https. If we try to talk to a non existent server we would get a
reply saying port not available. Destination port is also mentioned in the
client machine.\\

A transport level session is defined by source IP, destination IP, source port
and destination Portand the protocol, that is, TCP or UDP. Transport layer is
an end to end protocol because it is not changed from source to destination.

A socket is a (virtual interface) comunication endpoint.
\section{Different Terms}
\begin{itemize}
\item
Bits per second or hertz
is the frequency measured.
\item
How long does it take to reach from one end to the other
for a packet
is called $Packet$ $delay$ or latency.\\
\item
There is a round trip time which is the time taken to reach a destination
and return to the source for a packet(like a ping).
\item
Bandwidth Delay Product(BDP) : It is the product of bandwidth and the round
trip time.
This is measured in bits. A simple protocol for ensuring that all packets are
sent
is to resend all the packets from the dropped packet after waiting for the
particular packet for a certain amount of time. This is not efficient because
if after the dropped packet many packets are sent all these will be dropped
and present.
\item
Delay components is the time taken between the number of different routers and
such in betweeen.\\
\end{itemize}

No of routers in between = 4.\\
physical length : 100 bytes.\\
Link bandwidth : 10 Mbps.\\
Link length : 10 Km and optical links are used.\\
Speed of light in fiber : 2*$10^8$ m/s.\\
1 km takes 5 micro seconds.

Propogation delay is the time taken for one bit from source to destination.
Therefore time taken in this case is 4*10*5 = 200 micro seconds. On
one single link is 50 micro seconds.\\

Transmission time = Packet length/ Link bandwidth = 100*8/10*$10^6$ = 80 micro
seconds
per link. Therefore total of 320 micro seconds.\\
Basically the first bit will reach at 200 micro seconds wile the last one will
reach at
520 if only these are there as these are streamed not sent one by one.\\

Processing time (just assume it to be 50 micro seconds for now):\\
Verify error checksum.\\
Extract IP destination adder.\\
Lookup in the routing table.\\
Update header and recheck.\\

Generally it is 32 ns to process a packet.\\
If a link is busy recieving a packet, the other packets are buffered.
This gives an extra delay. We need to take into account
average queueing delay.
\section{Physical layer}
\subsection{Circuit Switching}
Every voice call requires 64 kbps and we multiplex different voice packets.
The different inputs to a multiplexer work at 64 kbps and the output is at 1.5
Mbps.
TDM is like round robin. We have a frame with 24 slots and every frame is for
125 micro seconds. The number of slots in the frame is fixed. It doesnt look
at the number of active members. That is even if there is a free slot it wont
remove it from the queue.\\
Constant bit rate traffic is where we sample a certain fixed amount every second
(here 8000 frames per second.)\\
T3 has 28 such multiplexer lines coming in which corresponds to
28*24*8/(125*$10^{-6}$) (8 is for byte to bit conversion).
I am given 1 byte every 125 micro seconds.
28 T1 lines come into one T3 link. Each T1 link is for 1.5 Mbps as seen earlier
and hence it becomes 45 Mbps.\\
A circuit is an end to end path with dedicated resources allocated
on each link of the path. A switch can have several input links.
All switches have a forwarding or switching tables.\\
The table consists of Input interface, input slot, output interface and
output slot. Input interface corresponds to which link is coming in and
input slot corresponds to which input of the multiplexer comes here.
This table might keep changing.\\
Connections can get modified hence connections can be lost etc.
\begin{itemize}
\item User makes a circuit request.
\item
Cicuit establishment phase
\begin{itemize}
\item Find a path to use from S to D.
\item Find out if free slots are available (in every link along the path) to
transmit via it and reserve it till you do so.(check whether reverse is also
available)
\item Complete bidirectional circuit setup.
\end{itemize}
\item
Start voice or data communication.
\item
Cicuit tear down phase where we need to clear all the slots that are reserved.
\end{itemize}
If any of the links goes down then all the data goes down as they cannot be sent
from them hence the whole process has to be repeated.
Adding customers becomes very hard because then we need to add switches each
time when number of slots are exceeded.\\

I might not want 64 kbps on one slot rather I might want more
slots but would want to restrict within 64 Kbps.

Circuit  switching is a layer 3 protocol.

\subsection{Packet Switching}
\begin{itemize}
\item There is no circuit setup.
\item Every packet has a source and a destination address
\item Packet is forwarded on hop by hop basis
based on the destination address.
When it reaches the first switch there would be no address information.
So for all these we add an address over this.
\item Design goals :
\begin{itemize}
\item Support for bursting.(cycle of activity and silence)
Variable bit rate support is not easy though.
\item Resilient to failures. Because now if there is a link cut somewhere
then this packet will be set via another path. On the fly we can
find the path instead of fixing it before hand. Basically the forwarding
table is modified when a link goes bad. Loops can occur.
\end{itemize}
\end{itemize}
Features of Packet switching :\\
\begin{itemize}
\item Every packet has a source and destination address.
\item Each packet of a flow is routed independant of  other packets
from the same flow.
\item Different routes can intoduce different delays making
in order delivery hard.
\item Buffers are present in a packet router.
\begin{itemize}
\item Store and forward routing.
\end{itemize}
\item Supports variable length packets(minimum 40 bytes and maximum
of 64 KB headers minimum is 40 therefore for a packet it is 40 + 1 = 41 bytes)
\end{itemize}

Based on the time stamp from the transport layer to the application layer
if underlying is TCP or RTP then it will convert it into in order by
time stamps.
There is one component which is queueing due to which there is
latency introduced. In circuit switching there is no buffering.
The latency components are :\\
End to End latency is made of these components\\
Propogation delay\\
Transmission delay\\
Processing delay\\
queueing delay is also there in packing switching\\
Router is a software based switch where it $\lq routes \rq$ the packet
(determines which router it goes to next etc) it recieves.

\subsection{Compare circuit and packet switching}
\begin{itemize}
\item A circuit from source to desination while the other is a
link to the next router.\\
In cicuit first packet has to wait for circuit setup.\\
\item In order in circuit switching while not in the other.\\
\item Packet doesnot contain source and destination address in circuit
switching while it does contain in the other.
\item Circuit doesnot support buffers while the other does.
\item Fixed length in cicuit and variable length in the other.
\item Circuit switching doesnot support efficient burst style of traffic while
packet switching does.
\item Buffer overflow can result in dropping packets
in packet switching while not in the circuit switching. However
packet switching and circuit switching are prone to bit errors in
transmissions as there is no correction mechanism.
\item Looping can result in dropping packets
in packet switching while not in the circuit switching.
\item Link/Node failures have large over head in cicuit switching while
the other has resistance to link/node failure.
\item Guaranteed bitrate in circuit switching and no such guarantees or
dedicated allocation in the other.
\item Rigid and inefficient use of resources while more flexible
and more efficient resource allocation or utilis
\end{itemize}
Circuit switching was used long back for telephone network.
Both circuit and packet combined to give virtual circuit
switching. ATM is one of it which means Asynchronous transfer mode.
Another one is Network Protocol Label switching. The aforementioned two are
virtual circuit switching methods.\\

LAN to Access Network. Access network is the last mile for the
network layer. Access networks are of two types.
wired and unwired or wireless.\\

Modem started at 300 bits/second and ended at 56kbps.
DSL which supports 2Mbps. There are cable modems and then
optical fibers.\\

Wireless are wifi and cellular. LTE, EDGE/GPRS, HSDPA and CDMA are
types of wireless networks.
\section{Link layer}
Transmissing bits over a physical medium. We characterize it
over max bit rate, bit error rate, cost, power consumed and
maximum distance possible.\\
Green ICT which refer to Information and communication towers
which tries to reduce power in wired networks.
Generally the layout of telephone network is by copper lines.\\

On a copper link for 1 or 2 kms we can go at 100 Mbps,
1 GBps (10's of metres) and 10 GBps(few metres) is even possible which
is inversely proportional to the distance.\\
If we want to build a long travel then we need to re amplify etc.
Bit error rates vary from $10^{-6}$ to $10^{-9}$ bit error rates.\\

Fibre optical cable is very thin strand of glass. It can send
upto 25 to 75 Tbps. Today we can support only like 10 to 40 Gbps
due to electronic limitations. Fibre can go to more than
100s to 1000s of km without regeneration of signal.
Bit error rate is $10^{-12}$ to $10^{-15}$ (extremely less).\\
Wireless supports a maximum of 100 Mbps to 1 Gbps.\\
1 Gbps is possible in 11ac. Bit error rate is approximately from
$10^{-3}$ to $10^{-6}$ is the bit error rate and cost is also high.
Power is more than copper which is about 100 milli watts to about
(1 watt for cellulars). Bluetooth is about 1 to 10 milli watts. NFC is
about 1 milli watt.\\

Distance is about a few kms. There is some health issues regarding
maximum power to be transmitted.\\
b is the Bit error rate.
Let a packet length be l bits. Packet error rate = $1 - {(1 - b)}^l$
Probability of succesful transmission is ${(1 - b)}^l$.

L = 12000 bytes = 1024 bits. b = 0.0001 : error is 0.69
which means it is bad enough.\\
\subsection{different TCP protocols}
\subsubsection{Stop and Wait}
If we have 100Km and 10 Gbps then efficiency turns out
to be less than 1\%.\\
Maximum through put is the sending bit rate which is simply
product of the efficiency times the link bit rate.\\
Max throughput = No.of bits/packet/$t_total$.\\
= 0.0119 Gbps in the above case.\\
Stop and wait will not work if the bitrates or the propogation
delay is huge.\\
Let bit error rate be b, for 1km/10 Mbps.\\
Let b = $10^{-5}$.\\
Case 1 :\\
Whenever we start sending a packet we started a timer and it is
closed when the ack is recieved or whether it receives an upper
limit. The packet will be re-sent when it doesnt receive any
ack packet.\\
Case 2 :\\
The ack is corrupted or not received then again
the packet is resent.\\
Case 3:\\
Timer times out but acknowledgement is sent from the receiver.
In this case we would have a duplicate packet. For every time
we properly receive a packet we would keep a counter on the receiver
which we would increment whenever it sends an ack. So when
we receive the something before this count then we drop
this packet.\\
If size is small then efficiency goes bad becuase of ack packets and such.
When huge packets are sent then it could be dropped because of bit error rate,
then again it is bad.\\
Geometric random variable :\\
p is the probability of success.
P(x = i) is success probability in the $i^{th}$ run.
P(x = i) = $(1-p)^{i-1}$*p.\\
expectation is 1/p.\\
packet error rate = $(1-b)**{size}$ where b is the bit error rate.\\
Probability of success for a packet is 88.40\%.\\
Average number of attempts = 1/88.4 = 1.131.\\
Avg time taken for succesful transmission = $t_total$*1/p.\\
Therefore Efficiency = Efficiency*p.\\
If bandwidth delay product(ratio of Packet size to speed of link) is
small then it is ok to use stop and wait.\\
Efficiency is approximately  $t_trans$/($t_trans$ + 2*$t_prop$).\\
Efficiency is approximately = 1/(1+2*proc/trans).
Let a = proc/trans. Efficiency is 1/(1+2*a).\\
Efficiency also depends on error probability which gives a trade off.\\
Try graph for BER = $10^{-4}$ for 10 Km and 10 Mbps\{1500 bytes, 40 bytes\}.
\subsubsection{The Sliding Window Protocol(Pipeline) - Go Back N}
The operation of the protocol is quite simple. We simply send all the packets
sequentially and dont wait for the ack packets. When all are sent we start
receiving. Transmission window is set of bits sent and not yet acknowledged.\\
I should keep sending until my first ack is received.\\
Efficiency = SWS*trans/total. For stop and wait SWS = 1.
If SWS is larger than one then it is pipelined.\\
Larger the window better the efficiency.\\
Ideal SWS = total/trans (greater than or equal to).\\
100 Km, 10 Gbps, 1500 bytes, 40 bytes, proc = 0.\\
trans = 1200 ns.\\
ack = 32 ns.\\
prop = 5*100 = 500 micro seconds.\\
total = 1001.232.\\
Ideal SWS = 1001.232/1.2 = 835 packets to be sent within this window.
Number of bits in sequence field will be 10 bits.\\
For this system if people offered only 8 bits then efficiency is 0.31.
TCP generally operates at 32 bit sequence numbers.\\
In a typical session there will be a wait time for packets to arrive
at the sender stage.
LFS is last frame sent and LAR is the last ack received.\\
LFS - LAR must be less than or equal to SWS.\\
Generally $SWS = RWS$ and rarely $SWS > RWS$.

Receiver window side = 1 implies we accept only that particular
sequence number we have been waiting for. In this case it is called Go back N.
LAF is the last acceptable frame and LFR is the last frame received.
initial is LAF = 0. In the next frame we make it 1. Every time
I get a sequence number we update it to the next sequence number
which is what is sent. In Go back N I will discard
every packet which comes after a packet which is dropped.
After that we repeatedly sent ack 2(which is called a duplicate) telling
that we expect 2. Sender will retransmit from where it left off.
If there is a duplicate ack the sender resends.\\

Sender retransmits from the first corrupted/lost packet that
is for which timeout occurs. Receiver doesnot buffer any out of
order packets. The sender buffer size is greater than 1 else
it becomes stop and wait.\\
Cumulative ack and selective ack are two types of acks.
When the number of bits in the sequence number is small,
we would generally reuse sequence numbers.\\
Cumulative acks acknowledges all packets upto and including
$x-1$. Adequate for Go back N. Even if acks are dropped
but a future ack is received we can continue in Go back N.
So receiver can with-hold acks and send one for a bunch.\\

Number of bits required in Ack is
the same number as the sequence number. Whenever we get a
packet we have an ack timer. Before the ack timer ends
if we send ack with the data packet then it is called
piggy backed Ack. If there is a data packet to be
sent then it is better to club ack with data in case of Go back N.
In piggybacked it is two way communication.\\

Window size and sequence number field size :\\
In Stop and Wait :\\
For stop and wait 1 bit sequence number is enough.
I send 0 the receiver is supposed to ack 1. To differentiate
between sending an already sent packet we use the extra bit in
sequence. Whenever we see a data packet we send an ack.\\

In Go back N :\\
Max SWS = $2^n - 1$ given an n bit sequence number.
Let n = 2, then we have 4 number of possible sequence numbers.
My SWS is set to 3 though so I wont sent 3 until I receive
ack for 0. After this I send 3rd.\\
SWS = $2^n$ is not a good thing because :\\
If SWS = 4, all 0, 1, 2 and 3 will be there.
Now if the ack sent back is dropped
then sender will retransmit all 0, 1, 2, 3 and this will
lead to a problem as receiver doesnot know whether it is
new or is it the old thing.
\subsubsection{Selective Repeat}
\begin{itemize}
\item Selective repeat has buffers and it can store out of order
packets.
\item Sender only retransmits droppped packets.
\item Cumulative ack should be replaced by selective ack/selective nack.
Only out of order packets expected in the current window.
\item Sender window and receiver window are in sync.
\item Receiver window size is greater than 1.
\end{itemize}
These are the main three protocols. There are many variants to this.
Max sender window size for an n bit sequence = $2^{n-1}$.\\
We will split the data packets into two windows for $2^n$. That is for 3
bit sequence, we have 0, 1, 2, 3 in the first window and 4, 5, 6, 7 in
second window. Basically between two consecutive windows there shuld be no
overlap of sequence numbers. Only when all the
packets in one window is sent, we move to the next window.
Reason why the windows should be non overlapping. Receiver
might not know which of the overlapped packets did it receive if its
ack was dropped in between.\\

\subsubsection{Efficiency}
Max efficiency = min(sws*$t_{trans}$/$t_{total}$, 1).\\
Efficiency for SAW that is Stop and Wait = P*$t_trans$/$t_total$.
Efficiency for SR that is selective repeat = P*SWS*$t_trans$/$t_total$.
Efficiency for GBM that is Go Back N it is somewhere inbetween.
\begin{itemize}
\item Under ideal conditions Go Back N is better because max SWS is higher in
it.
\item Under error prone situations there is no reordering
in GBN while reordering is required in SR but in SR only certain
dropped packets are $re-sent$ while in GBN we might have to $re-send$ the
whole window. So in such conditions SR is expected be better.
\end{itemize}

\subsection{Framing}
For every data sent there is a start of frame and there is an end of frame.
\begin{itemize}
\item Packet length info in the header. If length field is
corrupted then it becomes hard.
\item Byte stuffing. Put some flags. If flag appears
in a data, byte stuff adding extra data(escape sequence).
If escape also appears replace it by escape escape.\\
How long before error is discovered ? Find the frequency
of how much the escape sequence and then find the
average of how many times it appears in the data.
Length of flag / Length of data is the over head.
\item Bit stuffing is predominantly the same.\\
Flag is 01111110. Look at your data on a bit by bit basis.
This flag is used in the start and the end as seen before.
Inside the data part it will never have 6 consecutive 1's
appearing. Algorithm to make consistency is
replace 11111 with 111110.\\
Data : 01101011110011111011.\\
Transmitted data : flag 011010111100111110011.\\
This works because whenever we see five 1's we definitely
know that we should have a zero after it. Asynchronous
sent. There are synchronous things where we need to ensure
that there is periodic communication between sender and receiver.
\end{itemize}
Typical framing schemes which use both length field and
bit (or byte) stuffing.\\
Synchronous links : Send frames continuosly whether they contain
data or not. TDM is an example of a synchronous link.\\
SONET $Oc-1$ which is organized a matrix format which is 9x90
that is 9 rows and 90 columns. This would be one frame and hence
the value turns out to be 51.84 Mbps ($Oc-1$ would send one frame
every 125 micro seconds). $Oc-3$ would be 155.52 Mbps(3 times $Oc-1$)
and so on till $Oc-768$ which sends about 40 Gbps.
\subsection{Error Control Protocol}
Error control protocol is error detection and error correction which we do.
Given a data of size m bits add r bits, based on some
error protection function, that represent the error code and
send it to the receiver. Hence communication of $m+r$ bits.\\
Single bit parity :\\
Even parity scheme : In the transmitted bit stream,
the number of 1's is even. If it is odd add a one, if it is
even add a zero.\\
\begin{itemize}
\item This can detect when there are odd number of bit flips.
\item Probability of detection of errors is more or less 1/2.\\
\end{itemize}

$2-D$ parity is one where you take parity for every p bits
in the frame, add a (p+1)th bit which is parity bit. Now
for 1, p+1, 2p+1 ... add a parity bit and so on. Like row
and columns. Now for 1 bit errors,
we can locate the error and attempt correct correction as well.
If there is a 2 bit error we can detect where it is and sometimes
attempt correction but not always. If there is a 3 bit error,
(like 1 data bit and 2 parity bits) we might not even be
able to detect. Make the last bit available in (N+1)x(P+1)
as the parity of the number of 1's in parity column alone,
we can probability detect 3 bit errors. Four bit errors
are even harder to be detected. No of parity bits
is (m/p+p+1) which is O(sqrt(m)).\\

m bit message, r bit parity.
n = m+r. There are $2^{m}$ which are legal codewords
and others are illegal. Given $2^m$ set of strings, Hamming distance
of a code is the minimum number of bit flips across any two
valid codes. So for 2D this was 4. If the number
of bit flips is more than the hamming distance, then detection becomes
harder.\\

\begin{itemize}
\item If my code should detect all errors till d bits, I need
the hamming distance of two codes to be d+1.
\item To correct d bit errors, we need hamming distance to be 2d+1 bits.
That is if the code is closer to one change it to that.
\end{itemize}

New Protocol :\\
All error detection bits are generally placed in powers of 2 positions.
Bits in positions multiple of 1's will contribute to parity at 1.\\
Bits in positions multiple of 2's will contribute to parity at 2.\\
Bits in positions multiple of 4's will contribute to parity at 4.\\
and so on. In this case one bit errors can be corrected(similar to
prisoner wine question). Minimum number of parity bits to correct
all one bit errors is log n. No of
possible bit strings is $2^m$. To be able to correct
all one bit errors, I need to have m parity bits. For
each valid code word number of valid codewords that have
exactly one bit flip = n. Therefore number of
strings which are either correct or differ by 1 is
$(n+1)*2^m$.\\
$(n+1)*2^m \le 2^n$.\\
n = m+r and hence minimum r is log n.
m = 64kB = $2^{19}$ bits.\\
min r is 20.\\
Error correction is done in case of optical links in addition
to error detection.\\

Cycle Redundancy Check (CRC) :\\
\begin{itemize}
\item Polynomial arithmetic modulo 2.
\item 1 0 1 1 $\leftarrow$ $x^3 + x + 1$.
\item addition and subtraction is same as xor.
\item Define a generator polynomial c(x) of degree k.
The constant term is 1. Other coefficients can be 0 or 1.
\item Given M(x), a message of length m bits,
T(x) = $x^k$*M(x) which is essentially left shifting by k bits.
\item Let Rem(x) be remainder polynomial obtained
after dividing T(x) by C(x).
\item Therefore what is sent is $x^{k}M(x) - Rem(x)$.
\item Let Recv(x) be the received polynomial.
Receiver algo : If c(x) evenly divides
Recv(x) then no error else error present discard packet.
\end{itemize}

Example :\\
c(x) = $x^3+x+1$. M(x) = $x^7 + x^6 + x^4 + x^2+ x$.\\
$x^k$ = $x^3$ in this case $\implies$ $x^10 + x^9 + x^7 + x^5 + x^4$.\\
Now do bitwise xor division that is instead of
subtraction use xor operator. We will get remainder to be 111.\\
Tranmit(x) = 11010110000 xor 111 = 11010110111.
In the receiver verify whether remainder zero. After
verifying it, drop the last k bits and the rest is your
packet.\\

Analysis :\\
Transmit(x) xor Recv(x) is the error polynomial E(x).\\
\begin{itemize}
\item Single bit error :
$E(x) = x^i$.
Rem(E(x)/c(x)) = Rem((Transmit(x) xor Recv(x))/E(x)).\\
Hence Rem(E(x)/c(x)) = Rem(Recv(x)/E(x)).
If $Rem(E(x)/c(x)) \ne 0$, then error detected as
Rem(E(x)/c(x)) = Rem(Recv(x)/E(x)).
x/c(x) will always result in a non zero remainder if the first
and last coefficient of c(x) is non zero.
\item Two bit errors :\\
E(x) = $x^i + x^j$. $i > j$.\\
$Rem(x^i/c(x)) \ne 0$ and they are co prime
to each other.\\
There are some special generator polynomials that
will not evenly divide $x^l + 1$ for some known large values of l
(like till l it wont divide).
This l puts a limit on my packet length for l = 32768.
\item
Ensuring that error is detected : Coefficient of $k^{th}$ term
of g(x) = 1 implies we let coefficient $x^0 = 1$.\\
$x^{15} + x^{14} + 1$ doesnot evenly divide polynomials
of the form $x^l + 1$ where $l \le 2^15$.\\
Odd number of bit errors, if c(x) has x+1 as a factor then all odd number
of bit errors can be detected.\\
c(x) = (x+1)*c'(x).\\
Let E(x) be divisible by (x+1).
Then E(x) = (x+1)*Q'(x).\\
If odd number of terms are there in g(x) then it is definitely
not divisible by x+1 (factor theorem).\\
\item
Bursting errors :\\
E(x) = 0....0 111....000 meaning in betwen many went wrong continuosly.
CRC can detect all bursty errors of length less than or equal to k.\\
\end{itemize}
Sender :\\
Internet checksum : Organize the messages as 16 bit words
and add all 16 bit words using 1s compliment address. Let this sum be s.
Check sum = one's compliment of S.
Receiver :\\
all these three are repeated and you should end up with zero to
succeed.\\

\subsection{Transport Layer}
How is protocol design for L4 different from or similar to that
of L2(LLC) ?\\
\begin{itemize}
\item
Both do multiplexing and demultiplexing on the packets.
\item Provide reliable transmits over the corresponding lower
that can be unreliable. The UDP has SRC port, destination
protocol, check sum and length of the packet.
\item RTTs are much higher in L4 compared to L2 and design
becomes more complex.
\item RTT variance is very high and timeout estimation
is harder.
\item L4 has to deal with packet errors and packet drops
in intermediate routers.
\item L4 might choose to deal with congestion in networks.
\item L2 and L4 complement flow control(fast sender and a slow receiver).
\end{itemize}
\subsubsection{UDP}
Applications :\\
LAN Apps, video (RTT on the top of UDP).\\
Real time protocol has time stamp and a sequence number over
the underlying UDP.\\
Remote procedure call(RPC). Very small over head in UDP.
Need for low latency, low complexity and low overhead.
It is useful when the applications are more error hard.
Transmit window = min(SWS, RWS) which is called window based
protocol.\\
\subsubsection{TCP}
\begin{itemize}
\item Connection oriented, reliable transmission, byte stream, in
order delivery.
\item TCP has three phases : Set up phase, data transfer, release phase
which is called 3 way handshake.
\end{itemize}
TCP header has n bits which has source and destination
port number 16 bits, then it has a sequence number,
ack number. Every byte has a unique sequence number.
The sequence number sent is the starting byte's sequence number.
The ack number is a piggy backed ack in the sense that the source expects that
particular ack number which has no relevance to the sender's sending bytes.
Then we have window size of 16 bytes and check sum of 16 bits.
Options are $10-40$ bytes. Header bits = 4.\\
If only the header is sent even then the number of bytes sent is 4*x
where x is the value stored in the header bits.
Legal values for header is minimum of 5 as the minimum
sent must be 20 bytes. Hence if we get these 4 bits value to be less than 4
we know that it has been corrupted.\\
Check sum is computed on (entire TCP header + Data + Part of the IP header
+ sender address + destination IP address + protocol field). That is we
fragment these into 16 bits each and compute check sum for each of them.\\

$Applicaton \leftarrow TCP|App \leftarrow IP|TCP|App$, the maximum
an IP can send is $2^{16}$ bytes. Ethernet can send atmost 1500 bytes.
Hence IP will fragment here. We will take the packet and fragment
it such that we will send IP header with every fragmented packet and
we will not send TCP header from now on. Now there would some
fragmented offset which would be assimilated in the IP layer
of the receiver to reconstruct the packet. There are the source id's,
destination id and some kind of id (for reconstruction).\\
There might be further fragmentation. If a fragmented part is dropped
the whole packet will be $re-sent$. IP will have length of data not TCP.\\

$UAPRSF$ is acronym for Urgent Ack Push Reset Syn Fin.\\
Urgent pointer refers to the first bits which should be processed. Urgent is
assigned to zero if you simply start from the beginning of the packet,
one if the. Ack bit refers to whether there is a valid Ack. If it is 1 then
it means there is a valid ack field. There is no packet sanctity in TCP.
Its simply based on the number of bytes sent. If Push is set to 1 means we
need to send data in this packet immediately to the layers below or above
depending on whether in receiver or sender. Reset is a master switch
which rolls back to a particular point and starts $re-sending$ from there.\\
Syn corresponds to connection setup and fin corresponds to
connection finished.\\

3 Way handshake :\\
First we send Syn x, then receiver sends back Syn + Ack x+1 for which we send back
Ack. Default tcp uses go back n and cumulative ack.
If the receiver didnot receive the Syn then no issues because of timeouts.
Similarly all timeouts will be handled. Starting sequence can also be
clock based. If you receive a Syn + Ack z+1 different from x+1 then we simply
ignore this packet because we know that is not corresponding to it.\\
FSM for TCP :\\
Server :\\
Closed on listen goes to Listen.\\
Listen on (Syn) goes to Ack wait and sends back Syn + Ack.\\
Ack wait on Ack goes to Established.\\

Client :\\
Closed on (connected or syn) goes to (Syn + Ack) wait.\\
Syn + Ack wait on (Syn + Ack) goes to Established and sends back Ack.\\

Many timeouts are there on either side.\\

IETF : Internet Engineering Task Force.\\
(Jon Postel, Sep 1981).\\
RFC : Request for comments.\\
RFC 5281 is for congestion control.\\

Connection Finishing :\\
First Fin is sent then ack is received and then recevier sends back
Fin + Ack and sender sends back ack and closes.\\

Congestion Control :\\
This is based on SWS, RWS and CWND. These give an estimate of network's
capacity. Transmitted WS = min(SWS, RWS, CWND).\\
Water river gold example.\\
\begin{itemize}
\item Across connections we fix maximum segment size(MSS).
\item Negotiated between two TCP end points. Normally
during connection establishment.
\item An IP router cannot advertise an MTU(maximum transfer units)
less than 576 bytes.
\item Min MSS = 576 - 20 - 20 = 536.
\item MAX MSS = 64kb - 20 - 20 = 65496.
\item MSS is setup using the TCP options header.
\item Assumptions : RWS = 32KB and MSS = 1KB.\\
CWND = 1 MSS.\\
Max transmitted bytes in a window = min(SWS, RWS, CWND).\\
Therefore initially it is 1 MSS = 1 KB.\\

For each ack received before timeout, CWND is increased proportionate to no of
bytes acked. TCP segment number to window size
will be growing exponentially till it reaches RWS(if there are no
problems, that is, all acks are good). This is called K slow start phase.\\

When the first time a timeout takes place, we go back to 1 MSS
and we start again. Now the threshold will be set to half of before
and we go exponentially till there and from there we go linearly until
we reach maximum RWS.\\

TCP assumes that the only possibility
for packet drop is congestion. There is another methodology
called Threshold which is set as half of previous CWND.\\

\item Additive increase multiplicative decrease (AIMD).
\end{itemize}

UDP has a better throughput because there is no congestion control.\\
In TCP, there is dumb bell topology. Packet drop occurs
at common link. DCCP is a new alternate to UDP with control
mechanism.\\

TCP flavours :\\
\begin{itemize}
\item TCP Tahoe in BSD tahoe and supports fast retransmission.
Send only the packet for which 3 duplicate read.
\item TCP Reno which supports fast recovery.
\item TCP uses cumulative ack by default.
\item TCP buffers as much as the reciever window size specified.
\end{itemize}
We will have a hybrid of GBN and SR.\\
Look up silly window syndrome.\\
Nagle's Algorithm says that receiver should not advertise very small window
size. The least window size which you want to advertise is the MSS.\\
Timeout estimation :\\
\begin{itemize}
\item $Estimated RTT = \alpha*EstRTT + (1-\alpha)*SampleRTT$\\
$Timeout = 2*EstRTT$
\item
$Diff = EstRTT - SampleRTT$\\
$EstRTT += \delta*Diff$\\
$Deviation += \delta*(|Diff|-deviation)$\\
$Timeout = \mu*EstRTT + \phi*Deviation$
\end{itemize}

TCP options and extensions :\\
Type | Length | Options Info.\\
Type is of 8 bits and length is of 8 bits.
\begin{itemize}
\item MSS option which is setup using type 2. It is of length 4
(that is 16 bytes which includes Type, Length and Info) and option info which
specifies MSS size which is of 16 bits.
\item SACK permitted. The length field is for two bytes and has no
options info.
\item
TCP can support only maximum upto 512 Mbps due to its
window size constraint because window size is only of $2^{16}$ bits.
Window scaling is done so as to accomodate this. Max window size
is $2^{30}$. Length is 4 bytes. If sequence number is constraining,
simply add the timestamp to the sequence number and make it 64 bits.\\
\end{itemize}

Addressing and routing :\\
IP header format :
\begin{itemize}
\item 20 to 60 bytes.
\item 32 bit end point address, hierarchical address space.
\end{itemize}
Mentioning various parts of the header.\\
It has version bits, and B-bit offsets, IP field, D bit,
type of source, source address, destination address, options, checksum,
protocol bit specifying which protocol is used.\\

Max TTL = 255.(Interpreted as top count)\\
Every router formatting a packet will decrement TTL by 1.
Update TTL, checksum and then forward it to the next router.\\
if(TTL == 0)\\
Router is expected to send an ICMP error message ("TTL exceeded").\\
Internet continued message protocol.\\

IP also has a protocol field which refers to whether TCP, UDP
or whatever is used. If the data packet is corrupted then TCP
would be dropped. The D bit is set to 1 means do not fragment this
packet. Directive to a router processing the packet.\\

Find out maximum transmission unit along a path :\\
MTU of a path = min MTU of all routers along the path.\\

Ipheader :\\
Fragment 1 offset = 0 : mf = 1, Len = 1480+20.\\
Fragment 2 offset = 185 : mf = 1, Len = 1480+20.\\
Fragment 3 offset = 370 : mf = 1, Len = 1480+20.\\
Fragment 4 offset = 555 : mf = 0, len = 60+20.\\
mf indicates whether it is the end of the packet or not. If it is 1
it is not if it is zero it is.
If mtu is less than the size then further fragmentation
takes place. \\
\begin{itemize}
\item Defragmentation takes place at destination node.
\item If fragments arrive out of order, buffer them and
discard after some timeout if missing fragments dont arrive.
\end{itemize}

Options :\\
\begin{itemize}
\item Source routing which has a list of specific router and IP
address along the path. Used for diagnostic purposes.
\item Loose source routing. This has a list of intermediate landmarks.
There is no one fixed path. Used for diagnostic purposes.
\item Record route option : Each intermediate router processing the packet
adds its IP address to the options field of the packet.
Used for diagnostic purposes. There is a timestamp option as well
which can be used to track when this packet was created etc.
\end{itemize}

Ipv4 address dotted decimal notation. The 32 bit address
are grouped as 4 8-bit groups. MAC address is of 48 bits.\\
IP address has two paths :\\
\begin{itemize}
\item Network prefix.
\item Host prefix.
\end{itemize}

172.16.* to 172.31.* are private routable address.\\
169.254.* is a link local and not routable.\\
Classes of IP address :\\
\begin{itemize}
\item Class A :\\
It has 8 bits for the networks part and 24 bits for the host part.
The first bit is assigned to zero for this type. 0 and
127 are not supposed to be used. Therefore valid addresses are
$2^{8-1} - 2 = 126$. Max No of hosts per network is $2^{24}$
excluding some special cases like before. This is generally used
by the government.
\item Class B :\\
It has 16 bits for max number of networks where
the first two bits are set to 1 and 0 resepctively and 16 bits
for hosts per network. Max number of networks is $2^{14}$.
Max number of networks is $2^{16}$. This is for universities.
\item Class C :\\
There are 24 bits where the first 3 bits are set to 110(like options).
Max no of networks and maximum hosts per network
is $2^{8}$.
\item Class D :\\
This is called as Multi cast and first four bits are 1110.
\item Class E :\\
This is expected to come around in the future. The first four
bits are 1111.
\end{itemize}
We had the issue of lot of unused addresses (problem of averaging).
Therefore we were given some non contiguos class addresses.
Routing will take slightly more time. Number of addresses the router
has to lookup in the forwarding table again blows up.\\
The above methodology is called classless inter domain routing (CIDR).\\

Example :\\
Class A : 11.* to 11.0.0.8 we can split it into 4 networks.\\
The split up can be :\\
11.0000 ...\\
11.0100 ...\\
11.1000 ...\\
11.1100 ...\\
Hence network access is 10 bits. That is 8 bits before
and 2 bits now given by the split mentioned above.\\
We need to build a multicast tree. It is a seperate topic in itself.\\
No assignments regarding multicast !!!!!\\
Ipv6 talks about 128 bit address space. Here the address is normally
divided into two parts. Network part has 64 bits and Host is of 64 bits.\\
Link local :\\
Global unicast. How do we embed Ipv4 address in an Ipv6 address ?\\
32 bits can be embedded in the available 64 bits space. In the host
part we can embed the 48 bit MAC address.\\
ARP is the address resolution protocol. MAC id given for IP address
during broadcast protocol.\\
RARP : When I know the mac address how to find the corresponding IP address.
This is normally used when booting up. The rarp server will tell this guy's
IP address.\\
Ipv6 header :\\
fixed length.\\
First 32 bits : 4 bits | 8 bits service field | 20 bit flow label.\\
Next 32 :  Payload length (16 bytes) | Next header (8 bytes) | Hop limit (8).\\
Next 32*4 : 16 bytes (Src).\\
Next 32*4 : 16 bytes (Dest).\\
Below there is the Payload.\\
There is no protocol field, no checksum, no fragmentation. Find the MTU
and send correspondingly.\\
This highly depends on the other layers around this.\\
Next header corresponds to the options header (or) Protocol type.\\
After payload there will be an option header. Each option header will
also have a length header field. The lat option header's next header will
contain the protocol header of L4 protocol. This protocol is like
a null for the linked list of headers. The protocol header are preknown
values and below this there will be the data.\\

D-V (RIP) : A node's entire routing table is exchanged. Exchange with
neighbours only.\\
L-S (OSPF) : Link stat input of a node's neighbours. It can exchange it
with other network nodes.\\
Given a connected graph, writing the table is easy
just write the neighbours name and the edge weight corresponding to
it in the table. Now everyone gets their neighbour's routing table.\\
Consider nodes i and j and $C_{ij}$ be the current cost known from i to j.
We can continue computing the minimum cost easily by the above algorithm
and some easy techniques.\\
We get a routing table by this methodology and you can also
note that for this routing you need to go to this particular next neighbour.\\
These are all required for computing how to move in case of forwarding to
a particular node. All these will take some time for steadying. If any link
goes down, I will go ahead and recompute the shortest path
for certain nodes again.\\

The above is called bellman forde Algorithm.\\
Data plane happens in parallel to the routing

Effectiveness metric for a routing for a Routing algorithm for
a routing protocol: Convergence time for routing table building,
recovery time / updating after failures.\\
In Bellman Forde, updating at every step is atmost O($n^2$).\\
Time and space complexity of computations should be taken into account.\\
D's alg (Djikstra's Algorithm) :\\
Let Source be A, input graph and nodes, output shortest paths
from A to all other nodes.\\
For every node, there is a label (known cost form source, preceding node
to source).\\
Permanent :- Set of nodes for which shortest cost paths are
known at an intermediate step of alg.\\
Tentative :- $V - P$.\\
Initial :
Labels except A and Label for A is (0, A).\\
Hence P = \{A\}.\\
\begin{itemize}
\item Let i be the node most recently added to P.
\item Update labels of all the neighbours of i.
$C_{Aj} = min()$.\\
if($ C_{Ai} + C_{ij} < C_{Aj}$) Preceeding node of j is updated to i.
Now we find the smallest node and add it to P.\\
\end{itemize}
The above can be done with heaps and order E*log(E).\\

Sender packets are 64kB and ack packets are 40 bytes.\\
Control plane, Routing which gives duster plane and forwarding
of duster plance.\\

Congestion control, quality of service are of importance.\\
For a given flow how many buffers to allocate ?\\
Flow session is a 5-tuple : (Src IP, Dest IP, Src port, dest port, protocol).\\
For each flow allocate some buffer space.\\

When the packet arrives we no longer look at only the destination address,
some other parameters are involved.\\

Classification of packets :\\
Scheduling of plots is also based on the packet class. That is we classify them
based on the flow. The general scheduling algorithm is FIFO, weighted round robin,
shortest packet first. When there are different flows, how to share the sources ?\\
This is used like for fair sharing.\\
There are two ways of congestion control which is implicit and explicit.
Implicit means drop packets from tcp flows which would definitely reduce the
window size. Random early detection. If the average q length is less than some
threshold, go on else randomly drop a packet. Here the threshold is generally fixed.
This is on the receiver side basically. Explicit is Deebit which is old and
there is something else which is called explicit
congestion notification (ECN).\\

Explicit is by sending some bits(2 I guess) allocated to specify so as to reduce the
sender window size. Sender can accept these bits and change window size accordingly
or pack it.\\

Quality of service :\\
\begin{itemize}
\item Reservation request protocol.
\item Admission control protocol.
\item Quality of service scheduling.
\item Traffic shaping/policing.
\end{itemize}
\section{Router Protocols}

Circuit switching gives deticated bandwidth, but it is constrained by
inflexibility. Packet switching is more flexible. Virtual circuits switching(1990's - ATM networks).\\
Packet switching is also in the way of vritual switching. This is called
multi protocol label switching.

\subsection{VC switching}
There are 3 phases in doing this.
\begin{itemize}
\item Establishment phase (Set up of VC).\\
A connection request arrives : which says like
source, destination and some quality of service like 1 Mbps
connection, etc. Each such request goes to the controller.\\
\begin{itemize}
\item Determine the end to end path from s to d. This can be bidirectional
(ie) the to path's reverse can be used as fro.
\item Now we have a virtual circuit identifiers (VCID). Like VCID is 8 bits
long. VCID field is independent of the number of slots in a frame. We just
simply assign some identifiers to it in each node in the path.
Even if number of slots could be lesser, we are more or less fixing it. Now
this is only dependent on the number of bits in the VCID.
\item Using signalling messages, find a VCID that is free on each link of
the path.
\item Setup the forwarding table on each router along the path. Every
node has several interfaces. Every time the VCID is modified
before sending to the next router. Based on the packet number, the node
determines where it will be routed next to based on the VCID it receives
and routes it to that particular output interface of this router which
then will route it to the corresponding output VCID (ie) it would have
changed correspondingly change the VCID
to the VCID of the next router in the path. VCID is only link unique.
Hence we need interface id as well. There are input interface id's and output interface id's.\\
\end{itemize}
VCID bounds the number of connections possible on a link.\\

Let cumulative link capacity requests go up to $\alpha*LinkCapacity$ when
we require more than this. Since allowed max capacity is more than
the link capacity, we would now admit this connection here.\\
If the 1 Mbps is based on only burst traffic and is not
consistent, then we could use queuing accordingly to be able to
live up to the promise. In here a complete packet would be sent other than like
in circuit switching where packets are sent byte by byte.\\

\item Data transfer protocol.\\

Here we look at only the VCID but we actually dont need the
Src IP or the Destination IP address.\\
If a link fails, we need to signal source, and reestablish the
connection. Routing layer header need not have Src IP or Destination IP.\\
We can do aggregation at various levels so as to allow multiple
users connecting to one VCID. Some other routers will have large
bits VCID's.

\item Teardown phase.\\
This is the reverse phase of establishment.
\end{itemize}
ATM (Asynchronous transfer mode):\\
\begin{itemize}
\item Network layer tech to support Broadband-ISDV srvices. They wanted
to have 155Mbps connections.
\item Cells : Fixed length packet. US wanted 64 bytes, europe wanted 32 bytes.
They finally ended up at 48 bytes + 5 bytes header = 53 bytes.
\item Hardware switches.
\item VC switching based.
\end{itemize}
The header for this is : VPID(12 bits), VCID (16 bits) and
checksum for error corrections.\\
People started IP over ATM (IPOATM) which is like minimal ATM circuits
which are built over underlying IP. Tag switching which led to MPLS
are other protocols. MPLS is multi protocol label switching.
the label generally is VCID. It is similar to IPOATM, that is
underlying network is IP and hence it is called multi protocol. At
various nodes, we would have some special protocols
like IPMPLS(which may be ATM or something like that). The VC is
called an LSP (Label switched path). there would be some control nodes
which would transfer from IP to IPMPLS which would
add VCID to the path. Similarly there would
be another controller which strips the header
of VCID and go through as IP header alone.\\
The packet would now be like :\\
| L2 Header | MPLS Header | IPHeader | TCP |.\\
The MPLS Header is for 4 bytes of this.\\

\section{MAC Layer}
Point to Point Links. Shared/Broadcast Link(network)).\\
There is a common shared communication medium(channel).\\
MAC is medium access control.\\
The questio here is : When does the user get access
to the user ?\\
This is defined by the medium access control protocol.\\

Shared network topologies :\\
The bus topology :\\
\begin{itemize}
\item There is a bus.
\item The connections are called taps (like pings).
\item There is a terminator which absorb all the energies
and then there is no further transmission of the signal.
\end{itemize}
The star topology :\\
\begin{itemize}
\item Here there is a central server which is
a broadcast medium. Any frequence is associated to the server
and any signal from the servers would be broadcasted to
the nodes connected to the central servers.
\item This is used in Wifi hotspots on some fixed frequency.
\item Base station for cellular network.
Frequency division multiplexing (FDM)
splits it into individual channels and each channel
can be channelled into mulitple user based on time division
multiplexing (TDM).
\item Satellite. This was done by the Aloha protocol.
If you have a packet to send, send it. This is lightly loaded.
We will retransmit after some time. Stagger the users in time
based on some random variable. If it is not lightly loaded
then throughput goes down drastically.
This was done for connecting Hawaai as wifi was not enough and
wired was not possible.\\

Under light load, S = G where S is the throughput. But beyond a
certain load G, the throughput tapers of. If G is about 0.5 per unit time
it would be good.\\

Now let us look at Slotted aloha :\\
Time is slotted and unit length = 1 which implies fixed length packets.\\
Slot is the time taken to transmit one packet. TDM has a fixed frame structure
while this doesnot. A user will attempt to transmit a packet only at the
beginning of a slot. Every transmitter has a queue in which the packets
will be joined. Therefore if two packets collide we drop and ask to resend.
Every node must have some backup policy.\\
Let user x decide in slot no a and user y decide to send in slot b
after they actually collided if we assume that there is collision
between the users.\\

Under light loads, average packet delay[packet
generation to packet transmission] is less (slotting time).\\

Let x be the number of arrivals in time t. If X is a poisson variable
with parameter G where G represents the average number of arrivals per unit
time. then what is the probability that there are $x = k$ arrivals at the
time interval t = $e^{-Gt}*(Gt)^k/k!$.\\
In slottted Aloha, consider 1 slot and there are
3 possibilities : 0 transmissions, 1 transmission, 2+ transmissions.\\
1 transmission is succesful transmission seeing probability is $G*e^{-g}$.
Probability when X = 0 is $e^{-G}$. The collision probability
is $1-e^{-G} - G*e^{-G}$.\\

$S = G*e^{-g}$. Therefore $S_max$ will be around when $G = 1$ when $S_{max}$ is
1/e which is approximately 0.368.\\
Pure Aloha : Assume packet length = 1 slot. We will have something
here which would have optimum at G = 0.5 and the maximum will be 1/2*e which
is approximately 0.18.\\

Roughly $N*\lambda = 1$ works well.\\
Carrier sense multiple access (CSMA) : We are using wireless.\\

Similar to aloha and before sending a packet we sense the medium for
sometime.\\
When a node decides to send a packet :\\
Transmitter will sense the medium for some constant time.\\
If medium is idle for some predefined time, start sending the packet
else defer transmission and wait till the current transmission on the
medium is over (ie) keep listening till free and then go back to the
previous step.\\

Propogation delay and length of bus. This impacts network
performance. Before a signal reaches, I may send a retransmission.\\

If there is no collision and ack is received just continue
without changing anything.\\

Else retransmit packet at a later point in time based on some
backoff policy and start sensing the medium.\\

Utilization depends on bitrate, thorughput and propogation delay.\\

CSMA/CD :\\
Then people sarted designing collision detection as well.\\
Here we assume that there is something in the physical layer which
tells us whether a collision is detected. Else if collision
is detected, abandon transmission after sending some joining bits
and retry based on some backoff packing.

\end{itemize}

The ring topology is like a bidirectional link
where the users are connected in circular manner.
It was used in SONET for MAN (Metropolitan Area Network)
and in LAN for FDDI (Metropolitan Area Network).\\

MAC :\\
\begin{itemize}
\item Static access : Collision Free. This is like
only one person can talk at one point in time.\\

N users sharing a channel.
TDMA (Time Division Multiple Access). Basically
there is a notion of slot and then round robin
in that particular slot. Another thing is how
much time to allocate for each guy in the slot.\\

\begin{itemize}
\item Assume fixed length packets.
\item Define 1 slot time taken to send 1 packet.
\item Assume N is fixed.
\item Time on channel is divided into time frames each with N slots.
\item User i can send only in slot i, $1 \le i \le n$.
\item If User i has no packet to send, it is wasted.
\item We can also handle variable number of slots, we need
to have an extra controller in front of it.
\item Inflexible bus allocation.
\item Let $\lambda$ be the average number of packets generated
per slot.\\
Case 1 :\\
$\lambda$ very less than 1 (ie) lightly loaded system.
Waiting time to start transmission = 0 or $n-1$.\\
Avg waiting time is $(n-1)/2$.\\
Total packet delay = (N-1)/2 + 1 = (N+1)/2 which
is O(n) which is not exactly good.\\

When $\lambda$ is greater than 1, then it is heavily
loaded then each user always has a packet to send
across all timeframes. In one timeframe,
the number of packets sent could atmost be n in this
case which is nice in the sense, TDM makes these active.

\end{itemize}
\item Random access : Collision Based. This is
where everybody can talk even at once.
\end{itemize}
Ethernet is baed on CSMA/CD :\\
\begin{itemize}
\item Xerox started it in 1975.
\item DIX ethernet v1 was made in 1980.
\end{itemize}

IEEE 8023 was made in 1983.\\
8023 Format :\\
|Preamble | Destination MAC| Src MAC | Length  | Data          | CRC     |\\
| 8 bytes |    6 bytes     | 6 bytes | 2 bytes | 46-1500 bytes | 4 bytes |\\
Min packet length is 72 bytes and maximum is 1526 bytes.
Preamble is used for a synchronization of bits. That is we need to find
when we need to start accepting as bits from that port. This is for
matching the initial sequence of bits which indicates that
we need to expect a packet from that port.\\

In Dix ethernet there is a type field (which is actually the length field)
which indicates the higher layer protocol.\\
If length value is less than 1500 then it is the length and if it is greater
than 1500, it is considered as the type.\\

For ethernet, we have 500m ethernet segment. Atmost
4 repeaters between 2 hosts in an ethernet LAN.\\
It takes $T$ units of time to reach Z which is the next layer.
The minimum time that A should be sending till A hears Z's
first packet. Minimum packet length is the function
of the round trip time ($2*T$). Even if the last time slot
at which Z sends back to A, then A would be able to receive $2*T$.\\
Z would detect collision at T while A would sense a collision only
at $2*T$ that is when the last link sends back a packet which
would collide with the current sent packet.\\

Based on some math and calculation, they fixed the
length of the packet.\\
This was theoritically, with a lot of assumptions, computed to be
64.\\
A is maximum when p = 1/k, where A is the rate of success.\\
A = T/p.\\
Channel utilization : p/(p+2T/A) = 1/(1+54*a) where p is the probability that
this sends. Let p = 1200 micro seconds, 2*T = 512 microseconds.\\
This calculation is about 1200/(1200+512*2.7)
which is approximately 0.89.\\

As a increases, utilization decreases.\\
A increases when T increases or when P decreases which means transmision time
increases (ie) when bitrate increases.\\
Consider 1 Gbps ethernet, 64 bytes min contention slot
length and 1500 bytes packet length.\\
Util = (F/B) / (F/B + 2L*e/c). Then Util = 1/(1 + 2*(L*e*B)/(c*F)) if
we divide by F/B throughout. B is bitrate and F is frame size.\\
When bitrate is increased also everything good.\\
When bitrate is 1 Gbps and L is 2.5 Km. Assume c = $2*10^8$ m/s.\\
min packet length = 2TB which is 3125 bytes. Therefore we have to reduce
this and hence L has to be reduced (ie) the length of the bus
has to go down to accomodate proper bitrate.\\

Read up on bridged LANs before endsem.\\
Constructing spanning tree in a distributed manner.\\
Switched LANs (Ethernet) (Read about it in wiki) :\\
switching is based on the mac addresses. So this is not broadcast
and hence there is no need fro CSMA/CD.\\
Hub is a broadcast switch. This is where CSMA and CD come in.\\

\end{document}
