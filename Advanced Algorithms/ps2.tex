\documentclass[solution,addpoints,12pt]{exam}
\printanswers
\usepackage{amsmath,amssymb}
\usepackage[T1]{fontenc}
\begin{document}
\hrule
\vspace{3mm}
\noindent
{\sf IITM-CS6840-2015 : Advanced Algorithms  \hfill Given on: Feb 1}
\vspace{3mm} \\
\noindent
{\sf Problem Set \#1 \hfill Due on : Feb 15}
\vspace{3mm}
\hrule

\begin{questions}
\question
\begin{parts}
\part
Let the graph G be coloured using colours A, B, C, D.
Let D be the colour used to colour maximum number of
the vertices. Therefore $D \ge n/4$ (PHP). Consider
all the vertices coloured by $A \cup B \cup C$, say set S.
It is easy to note that the cardinality of this set
is less than 3*n/4. Any edge having a vertex coloured
with D, has the other vertex coloured with a different
colour (Graph is coloured properly). Therefore
S forms a vertex cover of G as any edge has atleast one
endpoint in S.
\part
A four clique is a graph which is 4 colourable
and planar and has minimum vertex cover equal to 3.
\part
We know that any optimum LP solution to vertex
cover is half integral.
Consider an optimum LP solution to graph G for the minimum
vertex cover. Let $a_1, b_1, c_1, d_1$ be the
vertices which are coloured by A, B, C, D respectively
and are chosen as 1 by the LP and their union be S.
Let |S| = k.
Similarly define $a_{1/2}, b_{1/2}, c_{1/2}, d_{1/2}$ whose union is S'.
Let |S'| = t.
We know that $k+t/2 \le mvc \le k+t$.
(mvc is size of minimum vertex cover).\\
Wlog $d_{1/2}$ be the one which is the maximum among
$a_{1/2}, b_{1/2}, c_{1/2}, d_{1/2}$. Now
we can easily see that any edge which has a vertex from $d_{1/2}$ as
one end point has the other endpoint either in S or S'.
Now consider the set $S \cup {S' - d_{1/2}}$ say S''. Any vertex
in $d_{1/2}$ has its other end in S'' because two vertices
in $d_{1/2}$ donot share an edge because of proper colouring.
Therefore S'' is a valid vertex cover. The size of S''
is atmost $k + 3*t/4$. Therefore $(k+ t/2)*(3/2) \ge k+ 3*t/4$
Therefore $(3/2)*mvc \ge |S''|$. Therefore we have a 3/2 approximation
for vertex cover when the given graph has a 4 colouring given.
\end{parts}
\question
\begin{parts}
\part
Let us show the 3 approximate solution.\\
Let us say $c_1$ to $c_n$ for the customers and $s_1$ to $s_l$
for the suppliers.\\
Let us build a set T which contains k suppliers finally.
\begin{itemize}
\item
First choose a customer and choose the nearest supplier to that
customer. Now T contains that supplier.
\item
Every customer will have a
distance to every element in S. The distance of customer is
defined as the smallest distance to some element in S.
Now at every step look at the customer who is farthest from the
current set of customers T (ie) maximum distance from the set
and add the nearest supplier of that customer to the set. If the
nearest customer has already been chosen, then choose any other
supplier from T and continue this step till you choose k suppliers.
It is easy to see that at no point the distance increases for
any customer. Let dist(a, b) denote the distance from a to b
and dist(a, T) denote the distance from customer a to supplier
set T.
\end{itemize}
Now lets prove that this is indeed a 3 approximation :\\
Consider the optimum choosing of these vertices to be
$a_1, a_2, ... a_k$. Now draw circles of radius Opt around them.
It is easy to note that all the customers fall within atleast one
of these circles.\\

Now we have two cases. At every step the fasthest customer we get
was from a different circle or that at some step we got
two customers from the same circle.\\

Case 1:\\
All from different circles.\\
Consider $c_i$ which belongs to the circle made by $a_j$ and which
we came across in our algorithm. Now we would have gone ahead and added
a supplier say b which is the closest to $c_i$ (or such a vertex
would have already been in the set). It is trivial to see that the
distance between b and $c_i$ is less than Opt because dist($a_j$, $c_i$)
is less than Opt but b is the closest to $c_i$. Every customer
in the circle is atmost at a distance of $2*Opt$ from this customer.
Hence distance between b and any customer in the circle is atmost 2*Opt + Opt
which is atmost 3*Opt. Therefore b covers all the elements in the
circle at atmost 3*Opt distance. Like this we can show an upper
bound of 3*Opt for all the circles and hence cover all the vertices
with cost of atmost 3*Opt. Hence the algorithm turns out to be a 3 Opt
in this case.\\

Case 2 :\\
At some point we choose two customers from the same circle.\\

Lets say the already chosen vertex was $c_i$ and the current vertex
with maximum distance from the set was $c_j$ and the closest one to $c_i$
is b. Now we know that the optimum value never increases in our algorithm.
Therefore optimum at this point of the algorithm is greater than or
equal to the optimum. The dist($c_j$) from the set S at that point
is atmost dist($c_j$, b) as $b \in T$. Hence the value
obtained by our algorithm is atmost the distance dist($c_j$, b) (ie)\\
Algo $Value \le Algo$ Value at $r^{th}$ $iteration \le dist(c_j, T)
\le dist(c_j, b) \le dist(c_j, c_i) + dist(c_i, b) \le 2*Opt + Opt \le 3*Opt$.\\
Therefore our algorithm yields a 3 approximation in this case as well.

Hence we find that in any case we get a 3 approximation algorithm.
\part
We know that hitting set is NP complete. Let us show
that if there exists a $3 - \epsilon$ approximation algorithm for
the k-supplier problem, then hitting set is in P.\\
The decision version asks whether there exists a hitting
set of size k.\\
Lets say the sets are $C_1, C_2,.... C_n$ and the
elements be $s_1, s_2, ... s_m$. Now consider the following
construction. $E(s_i, s_j) = 2, E(C_i, C_j) = 2, E(C_i, s_j) = 1$
if $s_j$ is present in $C_i$ and $E(C_i, s_j) = 3$ otherwise. Now
it is easy to see that the metric property is satisfied because
a triangle of side lengths 1, 1, 3 is not possible.\\
Therefore this can be seen as an instance of the k-supplier customer
problem where every $s_i$ can be seen as the supplier and $C_i$
can be seen as the customer. Therefore if there exists
an optimum k-supplier solution of 1 then we can be sure that
there exists a k size hitting set and if we get an optimum solution
of 3 then we can definitely be sure that there is no hitting set
of size k. We can also note that there exists no other possible solution
to this set up.\\
Let us assume there exists a $3 - \epsilon$ algorithm. Hence if
the optimum is 1 this algorithm can say atmost $3-\epsilon$ where
$\epsilon \ge 0$. Therefore in our set up, if the optimum
is one, then the approximation algorithm's output has to be 1 and
if optimum value is 3 then there has to be no hitting set of size k as
3/$3 - \epsilon \ge 1$ therefore if Algo says 3 then Opt cannot be 1
which implies there can be no hitting set of size k. Therefore if there exists
a polynomial time 3-approximation algorithm for k-supplier then there exists
a polynomial time solution for Hitting Set which is NP complete
thereby proving P = NP. Hence we are done.
\end{parts}
\question
Let us consider a solution where there is atleast one fractional
value. Else it is trivially true that all the corner points are
integral.\\

Let us say there are k edges which are assigned to 1 in that
fractional solution.Note that no two edges(which are assigned a fractional value
or 1) can be incident on any of these 2*k vertices because then the
equality constraint will be violated for that vertex. Hence this procedure
ensures that k vertices from either side was removed.\\

Now we are left with $n-k$ vertices on either side. Now each
of these vertices will have atleast two edges which have non zero weight,
because none of them had an edge which had weight as 1. Therefore total
number of non zero weight edges which would be chosen from this subgraph
would be atleast 2*(n-k). Now any graph which has more than number
of vertices - 1 edges definitely has a cycle in it(a fully connected
tree has only that many edges, so anything which has more number of edges
definitely has atleast one cycle). This graph has atleast
1 more edge than that and hence definitely has a cycle in it.
As it is a bipartite graph, the cycle can only be of even length.\\

Now let us say the lowest non zero weighted edge in this sub graph
has weight $\alpha$. Hence none of the other weighted edges
in this subgraph could have more than $1 - \alpha$ edge weight.
If there does exist an edge with more than $1 - \alpha$ weight, then
another edge emeanating from the vertex which is an endpoint of this
edge has an edge less than $\alpha$ which is not possible as none
of these vertices have an edge weight to be 1.\\

Now consider the even length cycle which is proved to be present
in the graph. Let us say that the cycle formed is $x_1, y_1, x_2, y_2,... x_l, y_l, x_1$ where $x_i$'s belong to the first partition and the $y_i$'s
to the second partition. Now traverse this cycle and reduce the first
edge by $\alpha$ and increase the second edge by $\alpha$ and so on.
That is alternatively decrease and increase the edges as you go around this
cycle. Now it is clear that for every vertex the equality is still
satisfied as each vertex has one of its edges increased by $\alpha$
and one decreased by $\alpha$. It is also clear that none of the
edges go below zero as in the original setting there was o edge
with edge weight less than $\alpha$. Hence this setting of
edges gives us a valid solution.\\

Now in the same cycle, alternatively increase and decrease the edges
by $\alpha$, that is increase the first edge by $\alpha$ and decrease
the second edge by $\alpha$ and so on. By similar argument to above we
can argue that it is a valid solution. Now it can be easily seen
that the original solution is the average of the two solutions
thus obtained(The set of edges increased by $\alpha$ in one setting
is decreased by $\alpha$ in another and vice versa).
Thus the original solution cannot be an extreme point.\\

Consider a 3 clique. It is trivial to see that this has no
mincost perfect matching. But if the LPP is applied we would
get all the edges equal to 1/2 as a valid solution, though
no integer valid point even exists. Hence this algorithm would not
work for non bipartite graphs.


\end{questions}
\end{document}
