\documentclass[solution,addpoints,12pt]{exam}
\printanswers
\usepackage{amsmath,amssymb}
\usepackage[T1]{fontenc}
\begin{document}
\section{Max flow}
f : V x V $\rightarrow$ R.
$\forall u \in V - {s, t}$ inflow = outflow. u to v is present implies v to u is not present (edge).
$0 \le f(e) \le c(e)$ is the capacity constraint.
Therefore total out of s - in of s = total in of t - out of s. Flow is a function.
We need to maximize this flow. Initial assumptions made were that there is one source
and one sink, no anti parallel edge and the vertices donot have constraints.
But each of the above can be reduced to the max flow problem.
Let us assume that vertices have constraints. Hence do the following :\\
Split the vertex with constraint into two vertices such that all the incoming go into one and
the constraint to that vertex is added as an edge from one split vertex to another and all outgoing edges from the other vertex.\\

\subsection{Ford Fulkerson Template}
Residual graph ith respect to a valid flow. (Basically the two constraints are capacity constraints
and inflow equal to outflow). $Capacity - Flow = Residual(C_f)$.\\
$C_f(u, v) = C(u, v) - f(u, v)$.\\
No antiparallel edges is assumed because now we are going to introduce a back edge of capacity
of the flow because using that we can $send$ back the flow.\\
$C_f(u, v) = f(v, u)$ if $u, v \in E$.\\
$C_f = 0$ otherwise that is there is no edge between u and v.\\
s, $v_1$, $v_2$, t.\\
Take a flow f in G.\\
Find a flow(f') in the residual graph by finding a path from s to t in the residual graph(R).\\
$f \uparrow f'(u,v)$ = f(u, v) + f'(u, v) - f'(v, u) if u, v is an edge.
Capacity constrains is trivially satisfied as we can see that only maximum of
flow can be sent back and maximum of residual can be flowed forward.
$f''(u, v) = f(u, v) + f'(u, v) - f'(v, u)$. As the flow is valid in the
residual graph we can similarly argue that the flow obtained by augmentation
is also conserved. Hence it is valid.\\
Our algorithm terminates when there is no path available in the residual graph.
That is repeat the residual network flow algorithm after you find the new flow
after first iteration and so on.\\
Every iteration is linear( O(m + n)).\\
Number of iteration can be utmost maximum flow which is feasible.\\
Therefore order is |max flow|*O(m + n).\\
These are called pseudo polynomial algorithms as it is dependant on some integer value.\\
$s-t$ cuts are those edges which if we remove the graph becomes disconnected.\\
There are exponentilly many cuts (s and some vertices and t and some vertices).\\
Capacity of a cut = sum over c(u, v) where (u, v) is a forward edge that is it
is from s cut to t cut.\\
net flow across the cut(s, t) = sum over all the flows from u to v (s to t) - sum over
all the flows from u to v (t to s).\\
Ford fulkerson method correctness :\\
s t cut is a cut where every vertex is either in s or t cut and the capacity is
defined as sum over all flows across the cut and the flow is the arithmetic sum
(maybe added or subtracted according to directions) of the cut edges.
This flow is exactly the same as the flow from s to t.
Flow conservation holds at every vertex. So all the flows within s will get cancelled
and similarly t. Basically the sum across the cut is the net flow.\\
Value of the max flow is equal to capacity of the min cut.
$max-flow \le min-cut$. For any s t cut we know that the value of the flow
is less that the capacity of the cut. Therefore the value of the max flow is less
than the capacity of the mincut.
\begin{itemize}
\item
If f is a max flow then there is no augmented
path in the residual network.(obvious as else you can pump more flow)
\item If no augmenting path in residual network then there exists a flow which has
value equal to that of the capacity of the cut. That is here all the forward edges are saturated
and all the reverse edges are removed.\\
Now put all the vertices reachable from s in S and rest all in T(meaning edges with non zero weight). Now consider this cut. Here if we consider the residual network of this graph
Now our claim is all back edges have weight zero and all the forward edges are saturated.
This is true because we cannot have more in the residual network as all these if they are not pushed into
t flow will increase but flow cannot be more than min cut's capacity. Hence we are done.
\end{itemize}
\subsection{Edmond-karp}
f = 0.\\
Construct $G_f$.\\
while there exists an $s-t$ path in $G_f$ :\\
Let p be any shortest path between s and t(unweighted).\\
Now augment using this.($f = f' \uparrow f''$)\\
And repeat.\\
m is number of edges and n is the number of vertices.
Now we already know that it is correct. Lets show that the number of times
an edge gets chosen is atmost n(number of vertices).\\
It is now ensured that every time the shortest path between two vertices is
bound to increase or remain the same.\\
The above can be proved by taking two cases as to where a new edge is
introduced in a path and a new edge is not introduced in the path.\\
When an edge became critical it definitely increases the path length
( by atleast 2) of some shortest paths between any source and some other vertex.
Therefore when each edge becomes critical(that is going to be reversed) then
the edge introduces a path difference of atleast 2 from the source to atleast one vertex. This particular edge $cannot$ be critical for more than n/2 times because the
path distance cannot be more than n between any two vertices.\\
Hence the number of iterations doesnot exceed mn.\\
Hence the above algorithm is polynomial.\\

There is an nXn grid m persons in the grid. These m persons should
reach the boundary in vertex disjoint paths -- Done

\section{P, NP}
From  optimization to decision problems.
A language belongs to NP if there is a certificate of polynomial
size and verifiable in polynomial time.\\
Co-NP is compliment of NP.\\
Stuff about NP completeness and reductions.

$VC \le_p Hamiltonian$.\\
\begin{verbatim}
For every edge u to v do the following :
    .     .
    |     |
    .  X  .
    |     |
    .     .
    |     |
    .     .
    |     |
    .  X  .
    |     |
    .     .
    X represents cross edge between 1st an 3rd.
    Now if for an edge if we choose only one vertex it is easy to see the cycle
    entering at u and exiting in the u side and not otherwise. If both are
    chosen then choose the straight path from u to exit and similarly for v.

    Connect adjacent edges that is by some bfs order. Now the only problem
    would be to make for entry and exit.
    Now have k selectors S1 to Sk which are connected to all the entries and exit.
\end{verbatim}
\section{Approximation Algorithms}
Algorithm of an instance is less than or equal to some $\alpha$
times the optimal answer for all instances. We would be happy if
we could get $\alpha$ as close as possible to 1. Even for some polynomial
time solvable problems we might go to find approximation algorithms.

We will find a lower bound and based on that we will guarantee
the above $\alpha$.

Lower bound :\\
\begin{itemize}
\item LB should be as large possible.
\item We should also compute the LB in polynomial time(efficient).
\end{itemize}

\subsection{Vertex cover problem (minimum)}
Maximal matching is one where you cannot add anymore edges to a
particular matching. Size of the maximal matching is less than
vertex cover. If you choose all vertices in a maximal matching
M pick both end points of edges in M to be VC. Therefore
Our algorithm get 2 times the size of the maximal matching
which is atmost 2 times optimum.\\

Worst case example for Vertex cover is a complete
bipartite graph. If Lower bound is fixed and there is a
possibility of set of equalities to the $\alpha$ then we
cannot find a better algorithm for that particular LB.

\subsection{Max Cut}
To design an algorithm where we get m which is atleast
$\alpha$ times the optimal where $\alpha$ is less than 1.\\
Partition vertex sets into two sets of vertices where the
number of edges between them is maximum.\\

Take an arbitrary vertex $V_1$ in A. At every point
next to this take a vertex $V_i$, put it in B if it is adjacent
to more vertices in A than in B and vice versa.\\
Let us show that the $\alpha$ factor is 2. Optimal can never
be more than the total number of edges. Now
how much ever we produce in the max cut it will be greater than
total number of edges divided by 2. This is because at every
stage we introduce atleast as much edges in the cut as within
the partition. Hence number of edges in the max cut is atmost
m/2.\\

Travelling salesman problem :\\
G(V, E) : Sum of the weight of the routes is minimum.
In its full generality TSP does not admit any approximation
algorithm.\\
Let us prove that $Ham \le TSP$.\\
G(V, E) to G(V', E') and C.
Keep vertex set same. Let us give an edge weight of one
to every edge present in G and a weight of 2 to every edge
in G'. Now set C = n. If there exists a tour of weight n
it implies all the edges chosen are of weight 1 and hence there
exists a Hamiltonian cycle.\\

No $\alpha$ approximation algorithm available :\\
This is because given any $\alpha$ to the problem, blow up
the cost of the un present edges to $\alpha$*n + 1. If G
does not have a hamiltonian cycle then any tour gives
a cycle of length atleast $\alpha$*n and vice versa. Therefore
given any such algorithm we can go ahead with the above reduction
and then apply the polynomial time approximation. Hence we solve
Hamiltonian cycle is solved in P.\\

Let us assume that sum of two edges is greater than the third
edge in a triangle. Then finding the MST which can be converted into
a tour, will give atmost weight of twice the MST which is atmost
the optimal (minus an edge).\\

Travelling Salesman Problem :\\
Cost of MST is less than or equal to OPT.\\
If triangular inequality is satisfied, then do the depth first
traversal from the root and instead of backtracking to the parent
go to the next guy in inorder traversal or more like it. This would
give the upper bound on the algorithm as 2*MST.\\

Christofide's Algorithm :\\
If we pair up some of the odd degree vertices we will reduce the
optimal value.
Pairing of odd vertices cost we would want it to be less than 0.5*Optimal.\\
We will find an eulerian tour and if we repeatedly short circuit, we will
end up having a proper TSP tour.\\
There are 2k vertices which were odd in our chosen MST.
There is an MST we have. Then we have a valid TSP tour and find all the odd
degree vertex and connect them all. Now there is a tour passing through
all these odd degree vertices for which we find a mincost perfect matching.
The cycle through this perfect matching is less than the optimal.
The mincost is definitely less than half the optimal value (as
compliment is also a matching which has atleast as much as the mincost).
The minimum now is 0.5*Optimum + MST which is less than 0.5*Opt + Opt
which is 1.5*Opt. Hence the final $\alpha$ we end up with is 3/2.

We can get an euler tour after the above steps and then short cicuit it
to get a hamiltonian cycle (travelling salesman problem).

Steiner Tree problem is one where given a graph we find the minimum
spanning tree on a subset of vertices where we may or maynot include
other vertices.

Possibly there are many steiner vertices. The leaves of any steiner tree
the leaves have to be required vertices.

Steiner set is compliment of required set.\\
If R = V and S = $\phi$ then it corresponds to a minimum spanning
tree problem and if |R| = 2 then it is shortest path between the two vertices.

When R can range from 3 to n-1 we will get that it is NP complete.
If we have an $\alpha$ approximation for Metric case then we will
have an $\alpha$ approximation for non-metric case.\\

We will first show a reduction to the metric case. We will find
all pair shortest paths between all the required vertices.
Now $C'_{ij}$ = shortest path between i and j.
$C'_{ij} \le C_{ij}$ where C was the original cost.
Now we also know that $C'_{xy} + C'_{yz} \le C'_{xz}$
because each of them are shortest paths. Hence it satisfies the metric
property. Let optimum Steiner tree in original case be $T^*$ and
current Steiner be ${T'}^*$. It is trivial that $T^* \ge {T'}^*$.
Now take any tree in current graph and replace all edges by their
corresponding paths. Now find minimum spanning tree for that sub graph.
It will have atmost the value of the tree found in the metric
case. Hence given any graph convert it into metric case,
find optimal(havent done yet) and do the back conversion to get an
approximation solution with the same $\alpha$.

Optimum for Metric case :\\
Take the subgraph on the required vertices and take
the minimum spanning tree on the required vertices(subgraph).
Take the optimal steiner tree. Now do a double tour on it
(euler as done before). Now weight is twice the optimal.
Now after this do a short circuiting. Now this will be
atmost twice the optimal due to metric conditions.
Therefore optimal for metric condition is atleast
half of that of the MST of the sub graph on required vertices.\\

$Alg(I) \le 2*(1 - 1/|R|)*Opt$.\\
This is because we are removing the largest edge
in the cycle we get from the double tour. Now
in that cycle we are removing an edge from the cycle.
The cost of the path is less than (1-1/|R|) times the
sum of the cost of the edges in the cycle. Hence the bound.\\

\subsection{Set cover}
There are S sets with total of T elements.
What is the minimum number of sets you need to choose
k sets of these S to cover all of these T ?
Now minimizing k is what is the set cover problem.
The problem can be a cost problem as well as in we can
add a cost to each set.\\

Out of the the overall subset of elements(T') in the current stage,
find the one with the maximum number of elements which are
subset of T' and choose it and now remove those elements from T'.\\

For every set, have a cost corresponding to it which is the
total cost divided by the number of elements not yet covered.
At each stage pick the least such set and proceed.\\

Cost of our algorithm is the total sum of all these costs.
Let $e_1$ to $e_n$ denote the order in which the elements got
covered.\\

Price of an element i is the effective cost of the set
that covers it for the first time. The price we
pay for each element is this order is non decreasing ($e_i$ to $e_n$).
The effective cost = cost of the set/number of uncovered elements.\\
Effective cost at no point can decrease hence e is
in increasing order.
Let us prove that $e_i \le Opt/(n-i+1)$.\\
If $e_1 \ge Opt/n$, then this means every element
$e_k$ is greater than Opt/n, this because. Give the cost
to each element in that particular order and arrange
the optimum in increasing order(a). Hence there
exists atleast one set with effective cost less than Opt/n
if this is not the case then, none of the cost of the element
is more than Opt/n. Hence there is a contradiction, establishing
price($e_1$) is less than $Opt/n$.\\
Formally, assume $e_1 \ge Opt/n$.
Optimum cost = (|$S_1$|/|$S_1$)*cost($S_1$) + ... (|$S_n$|/|$S_n$)*cost($S_n$).
which is equal to |$S_1$|*effcost($S_1$) + ... |$S_n$|*effcost($S_n$),
by our assumption this implies atleast one of them
had effective cost of $Opt/n$ as if all had greater, then this means
this Opt > Opt which is contradiction. Hence this establishes
a contradiction as we would have chosen this particular
set in the original algorithm in the first iteration.\\
Hence $e_1 \le Opt/n$.
Let us assume that k-1 elements have already be chosen.
Now we need to choose from n-k+1 elements. $Opt' \le Opt$
as covering a subset of elements we would use only lesser cost or
at most the same cost for covering this subset. We know
that $e_k \le Opt'/(n-k+1) \implies e_k \le Opt/(n-k+1)$.
Hence summation would be less than $Opt*(1 + 1/2 + 1/3 + ... 1/n)$
which implies our cost is less than Opt*ln(n).
F is the number of times an element occurs.\\

\subsection{Metric K-Center Problem}
We are given n vertices and distance between any two vertices.
We have to choose k centers such that maximum distance over the
minimum distance between a non center and a center is minimized.
Radius of clustering maximum over all such center non center distances.
Every point should fall into atleast one of the circles (circles
are drawn from the chosen set as center and radius as specified above).\\
\begin{verbatim}
Set S is empty to begin with
S = {i} where i is arbitrary
while(|S| < k) {
    j is the farthest point from all points in S
    S = S union j
}
\end{verbatim}
The distance would never increase. We want to argue that this
is a 2 approximation algorithm.\\
Let our set be S and optimum be $S^*$. Every vertex is assigned to
some cluster. Let the partition in the optimum be $V_1, V_2, ... V_k$.
Suppose our algorithm chooses one center from each of the optimal
clusters, then we are done because we can have the same partition and
at ever stage the maximum radius is atmost 2*r.\\

If we end up choosing 2 from the same set then it is trivial
that the maximum distance is not more than 2*Opt as then
if there exists another point which is at greater distance than
2*Opt our algorithm would have chosen it. Hence in either
case we end up with a 2 approximation algorithm. If we never
choose 2 vertices which are never off from 2*Opt then
it they obviously belong to different sets in the
optimal.\\

\subsection{Dominating Set Problem}
Dominating set problem(decision problem) : G(V, E) |S| = k.
For every vertex $v \in V$ either $v \in S$ or $\exists u \in N(v), u \in S$.
Is there a dominating set of size k. This is NP Complete. This can be solved
in polynomial time if there exists a $2-\epsilon$ approximation algorithm.
For every edge give weight of 1 whenever it is present and weight of 2 whenever it is not.
Now our approximation algorithm will tell us either 1 or 2. 1 means it is
a dominating set and 2 means not.
If there exists a $2-\epsilon$ approximation algorithm, then we can solve dominating set problem
in polynomial time based on its answer.

\subsection{Job scheduling}
Given n identical processes with different processing time. This
is called the length of the job ($l_1, l_2, ...l_n$). We are given
m processors and we have to distribute these m processed among the n
processors with minimum overall finishing time.\\
A decent lower bound would be $l_1+l_2...+l_m$/n. This quantity
is also known as the makespan of the schedule. Another is
MAX($l_1, l_2, .... l_m$).\\

Every stage allot a process to the least occupied machine.
$makespan(Alg) \le 2*Opt$.\\
Look at the last stage where you are about to add the last process.
Now at this window the processing time of the process which you are assigning
is less than Opt. By our algorithm, the time at which the last process was
allocated, is definitely less than the average of all the times. Now the time
for the last process is also less than Opt. As this is the last finishing job,
the time at which this finishes is the overall time. Start time
of this job is less than Opt and the time taken by the last job is less
than Opt. Hence overall is less than Opt. this can also be shown to be
($2-1/m$)*Opt.\\

Sort jobs according to processing time in decreasing order.
Now the factor reduces to 4/3 if we can assume schedule the last job
less than 1/3. Let the last job scheduled be $l_i$. Now discard the
other $l_{i+1}$ to $l_n$. Now Opt' for this instance is less than
original Opt. For this problem, that is the one where the last $n-i$ are
discarded, the least job which was the last
job scheduled. This is called LPT rule. Before scheduling last
the time taken is definitely less than Opt. Hence we try to show
that $l_i \le Opt/3$. Now the last job finished
is the makespan(Alg). If the last is less than Opt/3 we are already done.\\
If the smallest job in your instance has length greater than Opt/3,
then an Optimal schedule could have atmost 2 jobs on each machine.
Then we will show that this is $optimal$.\\
In the optimal, if any of them have more than 2 (ie) 3, then contradiction because
they have 3 with more than Opt/3. None of the processors
in which we schedule 1 will violate this. Let us assume that the second
schedule creates the problem. That is we have to now show that it never
violates and so on.\\

Let us assume that the $i^{th}$ machine when scheduling the
$j^{th}$ job exceeded opt. Now this job cannot be shifted to
any processor above it. It could be scheduled somewhere below
it. Hence at some point we would have a j' which is strictly
greater than j and such that it is not scheduled finally.
Hence we would probably schedule it in a unique processor
but this again leads to a problem because we need to remove
a process from a processor which has more length than even j'
and hence we are done. Hence this produces the best possible
solution.

\section{Local search}
\subsection{Max cut algorithm}
Number of edges which get cut is maximized in a partition.\\
Number of edges is an upper bound.\\
Now choose a vertex which when moved to the other set
introduces more weighted cuts. The algorithm definitely
terminates after m rounds because at every stage we increase
the weight by atleast 1 hence
the total running time is bounded by summation
over all edges times $n^2$. In unweighted we have $m*n^2$.\\
When we settle at a particular stage outgoing
weights is greater than incoming weights.\\
Sum of degrees within a set is less than the number of edges across.
Then we get that half approximation done.

\subsection{Better schedule}
Start with any schedule. Now move last job from this to
some other processor if it ends faster there. We move it to
the one with the least starting time. The processor time at which
a processor finishes the first will definitely not decrease in this way.
If a job gets moved twice, then when moving the second time,
the finishing time of this machine has not
changed to when it was before. This means that the definitely
not decrease part is violated at some point. Hence we will
never move a job twice. If it just remains the same till
this time then it doesnot
make sense to move this job to the other of same end time.
Hence a job gets moved only once.\\

\subsection{Edge colouring}
No two edges incident on the same vertex has the same colour.
$Opt \ge \Delta(max-degree)$. Given any undirected
unweighted graph you can colour it using $\Delta + 1$
colours. So we will see the $\Delta + 1$ approximation
algorithm for this. Still this is NP complete.\\
Take any edge and colour it using some correct
colour. Have a free colour set for every vertex.
When we colour an edge then the free colour set reduces.
We always have an invariant that the colouring is proper.
If we get $f_u \cap f_v$ is null, move to some
other vertex $v_1$ incident on u and see whether
there exists $f_u \cap f_v$, then colour it with that
colour.
\begin{itemize}
\item Pick an edge $u-v$. Let f(u) be the number of colours
unused in u.
\item If $f(u) \cap f(v) \ne \phi$, colour it
with some colour in $f(u) \cap f(v)$.\\
\item But if the same free colour is not available,
we will grow a fan on the vertex u as follows.
\item Choose a free colour from f(v) which would definitely be there
because we have $\Delta + 1$ number of colours(actually 2 free vertices).
\item If $f(u) \cap f(v_1) \ne phi$ for atleast one vertex neghbouring
to u, then we are done in the sense we will shift the colours
so as to accomodate all the edge colours.
\item if $f(u) \cap f(v_i) = \phi$ for all i.
There are d coloured edges on u. By php
we can show that two free colour sets of the neighbours
of u will intersect and that this is not missing on u
as if it is missing then we would have already
seen it in the previous case. See any free colour
on u ($c_i$). If we see any subgraph on $c_u$(red) and $c_i = c_j$(red)
(ie) the common colour.
\item The connected components on only the $red-blue$ edges
would give us either paths or even length cycles due to proper
colouring.
\item The root, i and j cannot sit in the same connected
component. From i if we have to have j in the connected component
it cannot be a cycle because both of them donot
have blue edges. But the root doesnot have a red edge, hence it cannot
be in that path. Therefore all these three cannot be on the same
component.
\item If i and j are on the same component, then flip
the colouring of this path throughout. That is make
red blue and blue red on this path. This is still a valid colouring.
Now we can do the shift colouring on u.
\item If the maximal path trails off, then recolour all those blue and
red as above and hence we will again be done.
\item The running time is O(m*n) to do this.
\end{itemize}

\subsection{Min degree Spanning tree}
Max degree of node in spanning tree is to be minimized.\\
Hamilonian path.\\
Algo :\\
\begin{itemize}
\item Start with some spanning tree.
\item $(u, v) \not \in T$ and $(u, v) \cup T$ completes
a cycle in T containing vertex x. Now remove the edge
incident on x which made the cycle (either of the edges as it is a
simple cycle). (u, v) path contains x.
\item $d_{T'}(x) = d_{T'}(x)-1$ and the degree of the other two
vertices might increase by 1. $d_{T}(u) < d_{T}(x)-1$ and
$d_{T}(v) < d_{T}(x)-1$.
\item Donot bother to make x the max degree vertex. Use all x
which has degree atleast $\Delta(T) - l$ where l = log(n).
\end{itemize}
$Alg \le 2*Opt + log(n)$.\\
$max(d_T(u), d_T(v)) \le d_T(x) - 2$.\\
Finally we will tell the max degree is atmost $2*Opt + log(n)$.\\
\subsubsection{Lower bound on Opt}
Take any spanning tree T and remove k edges from the spanning tree.
It will disconnect the spanning tree into k+1 components.
Every other spanning tree(including Opt) will have atleast k edges between
these different components else they would not be connected. Let these
edges that cross the cut be E' in the graph G. Consider the minimum
vertex cover of these k edges.\\
Any spanning tree requires atleast
k edges from E'. Now whenever you choose an edge from E'
you are increasing the degree of the vertex by atleast 1. Therefore
average degree of a vertex in $S \ge k/|S|$ where S is the minimum
vertex cover of all the edges of E'.\\
Any edges in E' not present in T will create a cycle which
has some vertex in the vertex cover this is because some vertex  in the
vertex cover will cover the back edge from these two components.\\
We stop when none of the $\Delta(t)$ to $\Delta(t) - log(n)$
can be reduced.\\
Let $S_i$ denote number of vertices with more than i degree in T.
There exists i such that $S_{i-1} \le 2*|S_i|$.\\
If we assume the contrary we would get that $S_{\Delta(t) - l}$
would be $2^l|S_{\Delta(t)}$ where l = log n which implies
it is greater than l and hence we would be done.\\

Let us consider that i. Sum of degrees belonging to $S_i$
is $|S_i|*i$. Therefore number of edges where one endpoint
belongs to $S_i$ and the other outside is atleast $|S_i|*i - (|S_i| - 1) = q$
which is $(i-1)*|S_i| + 1$. The maximum number of
edges in the tree where both belong to $S_i$ is $|S_i| - 1$.\\
Now we remove all these q edges. Now the vertex set $|S_{i-1}|$,
is definitely a vertex cover for the edges which were removed
and were one side incident (ie) $(i-1)*|S_i| + 1$,
because if it is not the case, then our algorithm wouldnot
have stopped. Sum of degrees of the vertices in that set
is $(i-1)*|S_i| + 1$. Hence the Opt is bounded
by $(i-1)*|S_i| + 1$/|$S_{i-1}$ which is atleast i/2.
Hence $Opt \ge i/2 \ge (Alg - log(n))/2$.\\
Hence $2*Opt + log(n) \ge Alg$.\\
We can easily note that this factor can be arbitrarily
reduced by choosing the required base for log.

\subsubsection{Polynomial Time Bound}
$\phi(T) = \Sigma(3^{d(v)})$.\\
$\phi(T) \le \Delta(T)*3^{\Delta(T)}*n$.\\
At every step, lets say degree of x, y and z change.
Therefore x reduces by 1 and y and z degree increases by 1.
Therefore the potential reduces by atleast $2*3^{d(x)-2}$.
Atmost $\Phi(T)$ could be $3^n*n$ and these give us the
final bound which gives order of $n^4$.

\subsection{Fully poly time approximation scheme}
\subsubsection{Knapsack}
Every bag has a size and a value. There is one bag
with maximum size into which you need to place
some of the bags there so as to maximize the value.
All $s_i \le B$. Sort it in max value/size and go
on. This can be shown to be arbitrarily large and would
fail. Assume a DP algo for O(n*min(B, V)) (knapsack).
We will use the O(n*V) algorithm.
We will scale down values of the current problem
into smaller values we are automatically done. But
we can live with only integral value, hence we
will round them off to the lower. After scaling down
we know that the pseudopolynomial is actually polynomial
and now we choose all the bags which we chose in the reduced
case.\\
I is the original problem and I' is the reduced problem.
I and I' both have same size B and $v_i' = floor(v_i/m)$.\\
Therefore $v_i'*m \le v_i \le (v_i'+1)*m$.\\
Alg = $\Sigma v_i$ for $i \in S \le \Sigma m*v_i' = m*(v_i')
\ge m*\Sigma v_i'$ in $Opt \ge \Sigma (v_i - m) \ge \Sigma v_i
- nm$.\\
Hence we finally get that Alg is greater than or equal to
$(1 - \epsilon)*Opt$.\\
If we choose m which is atmost $\epsilon*Opt/n$ which will give us
this bound.\\
We get the value of m to be $\epsilon M/n$.\\
$V' = \Sigma v_i' \le n^2/\epsilon$.\\
Therefore the running time is bound by $n^3/\epsilon$.\\
The dp way of solving knapsack is already known.\\

\section{LPP}
Many problems can be modelled as LP.
Set cover can be modelled as LPP.
The solution of this LP will give us a lower bound
on set cover.\\

Let f be the maximum frequency across elements of the set.
Given a feasible solution, if $x_j \ge 1/f$ we will
set to 1 and otherwise we will set it to zero. It is
clear that this doesnot violate the rule because atleast one
of the $x_j$'s corresponding to that element would be set to
more than 1/f. We also know that we increment each variable by atmost f.
Hence in each case if we increment it f times, (the worst case)
we end with f*Opt. Hence the answer we get is an f approximation.
This works for weighted set cover as well. This gives us a 2 approximation
for weighted vertx cover.\\

\subsection{Duality}
min : $7x_1 + x_2 + 5x_3$.\\
$x_1 - x_2 + 3x_3 \ge 10$.\\
$3x_1 + 2x_2 - x_3 \ge 6$.\\
Let $y_1$ represent first equation and $y_2$ represent
the second equation.
Now dual is :\\
Max : $10y_1 + 6y_2$.\\
$y_1 + 3y_2 \le 7$.\\
$-y_1 + 2y_2 \le 1$.\\
$3y_1 - y_2 \le 5$.\\
and $y_1, y_2 \ge 0$.\\

Lets say the minimization as primal and
the other as the dual. This is called the canonical form.
Weak duality : x = $(x_1, x_2, .... x_n)$, y = $(y_1, y_2, y_3, .... y_n)$.
The value of min primal will be greater than the value of the max dual.
This is easy to see from the constraints.\\
Strong duality : min primal = max dual and is optimal.\\
Complimentary slackness in optimal :\\
Either $x_j*$ is zero or the corresponding dual inequality is tight.\\
If these conditions are obeyed the it is optimum.
Hence it an if and only if.\\

If x* and y* are optimum then they satisfy complimentary slackness.\\
In optimal :\\
$\Sigma c_j*{x_j}^* = \Sigma (b_i*{y_i}^*)$
which is less than or equal to $\Sigma \Sigma a_{ij}{x_j}^*)*{y_i}^*$
and hence $\Sigma (c_j - \Sigma a_{ij}*{y_j}^*)*x_j \le 0$.\\
In similar way we can show that it is greater than or equal to zero and
hence we end up with exactly zero which ends up in proving one side
of complimentary slackness.\\

Only the corner points are candidates for optimum.

\subsection{Vertex Cover Again}
Consider vertex cover :\\
In VC, all corner points are either integers or 0.5 that is 0, 0.5 or 1.\\
Let us show that.
Any extreme point cannot be represented
as a convex combination of the polytope cannot be expressed as convex
combination of any two or more feasible points.
Hence there exists an $x_j$ such that it is not $0 < x_j < 1/2$ or
$1/2 < x_j < 1$. Let the first set be $V^1$ and the other be $V^2$.\\
$y_j = x_j - \epsilon$ for y in $V^1$ and $z_j = x_j + \epsilon$
for z in $V^2$. Similarly $y_j = x_j + \epsilon$ for y in $V^1$ and
$z_j = x_j - \epsilon$ for z in $V^2$. Hence we are done as
the average of these two points that is mid point of these two is the
original point and hence we are done. It is feasible as well
because of the way we are setting.\\

\subsection{Shortest path}
Let us find the shortest path between s and t.\\
minimize over all e in E $c_i$*$e_i$.\\
We can go over all cuts and make a constraint saying each of the
cost of the constraints has atleast 1. Any path is easily valid on
the LP. With any solution to the LP, we can go ahead and choose
the edges which were set to 1, which would finally give us
a path.\\

LP with exponentially many constraints can sometimes be solved.\\

\subsection{Prize Collecting Steiner Tree}
G(V, E) : You are given a graph with n vertices and e edges with a special
vertex r and cost associated with every edge and $penalty$ cost associated
with every vertex we leave out. The overall cost is cost of the tree and
the penalty incurred by leaving out some vertices.\\
Writing the optimization function is relatively easy and boundary constraints
are also easy.\\
Let us write a constraint for every set S which doesnot contain r :\\
$\Sigma x_i \ge y_i$ for all $y_i$ belonging to S.\\
This means the sum across these cuts is greater than 1 if
atleast one vertex is chosen and is greater than or equal to zero if
none or chosen anyway. This ensures connectedness from the root. Any
tree containing r is a valid solution and the minimum such tree we want
would give the solution here.\\

We know the maxflow mincut theorem that mincut = maxflow.\\
Now given a solution from an oracle with $(x_e 's)$ and $y_i 's$
all of them are 0 or 1,
For every $v_i$ do the following :\\

Let r be the s and $v_i$ be the t and start making the connections between s and
t as the original graph in itself. Now assign $x_e$ to every edge in this graph.
Now if the maxflow is greater than $y_i$ then we are done for all constraints
containing $y_i$ because mincut is greater than max flow. Similarly you
can show that if this is violated for any $y_i$, then there would exist
a cut which has a lesser value than $y_i$ which would give us a violation.
Hence this verification scheme goes through.\\

Now we have a fractional solution out of it.\\
Now choose every vertex with $y_i \ge \alpha$ (fixed later).\\
Now look at the mincost steiner tree containing these vertices
which can be computed with 2 approximation.

\section{N.S.Narayanaswamy}

\subsection{Max-Cut, Max-Sat (Vector Program)}
Sampling a random hyper plane :\\
Sample unit normals. This can be done by a normal
dostribution with zero mean.\\
Once this is done we can easily proceed for rounding approaches.\\
What problems are bad for this approach is what we will study.
Ratio between the integral optimum and the fractional optimum is
called integral gap.\\

Min vertex cover :\\
In LP formulation all corners are half integral.\\
Basic feasible solution corresponds to corners. Every coordinate
for vertex cover is either 0, 1 or half.\\

\subsubsection{Nemhauser Trotter}
You have $y^*$ which is the optimum for the vertex cover.
$y^*$ is the LP optimum. Consider $y_i > 1/2, y_i < 1/2, y_i = 1/2$,
as three different sets.\\

Claim : There is an integer $optimum$ that sets
less than 1/2 to 0 and greater than 1/2 to 1.
Now we have moved onto a sub problem which has only those which
are assigned to 1/2 if the claim is true. This is clearly feasible.\\

Proof :\\
Let us say some Integer optimum solution which assigns
$some$ less than 1/2 to 1 and $some$ greater than 1/2 to 0.\\
Let $D_1$ be those less than 1/2 set to 1 and $D_0$ greater than
1/2 to 0.\\
Claim : $|D_1| \ge |D_0|$. If claim is true, then we can simply
swap $D_1$ and $D_0$ that is take all $D_0$ into account and
leave out $D_1$. This would get us a better integer optimum.\\

Claim proof :\\
Assume contrary that is $D_1 < D_0$.\\
If this is true then do the following :\\
Increase all $D_1$ to 1/2 and reduce all $D_0$ to 1/2.\\
This is feasible and will be less than the original fractional
optimum. But this is not possible hence claim is proved.
From this we have a two approximation directly by setting all less
than half to zero and all greater than or equal to half to 1.\\

\subsection{Max cut}
2 SAT is in P (seen in ATOC).\\
Max cut is a special instance of max 2 sat. Is there a satisfying assignment that there are
k clauses ?\\

Look at the $2^n$ ways of partitioning it. Now we can see that
sum over all cuts of the edges is $M/2*2^{n}$ (because each
edge occurs either within the partition or across the partition with
equal probability).\\

Hence the average number of edges crossing the cut is M/2. Then
by conditional probability argument we can conclude that we can
choose the better average each time if we make a local move based on
an edge. Hence we have a half approximation here as then there exists
some value whose sample is m/2. The above process is called
derandomization.\\

\subsection{Max SAT}
Average is 7M/8 and we are similarly done via conditional
probability for MAX 3 SAT.\\
Let a clause have A positive literals and b negative literals
Pr(falsifying the clause) = ${(1-p)}^a*p^b$
therefore the average satisfaction is greater than (1-$p^2$)*m
for all clauses with greater than 2 literals.
If all unit clauses are positive, then we can
say that the same argument as the above would work.\\

If there are clauses which are negated but not those
which already appear then using renaming we can still go ahead
in the similar manner.\\
M/2 is guaranteed in the above methodology.\\
\begin{itemize}
\item
Max $\Sigma w_j$ which are weights for the clauses satisfied.\\
$Opt \le \Sigma w_j$\\
If the same variable occurs twice, then the following is true.\\
$Opt \le \Sigma w_j - \Sigma v_i$ where $v_i$ denotes min of $w_{x_i}$
and $w_{\overline(x_i)}$.\\
Probability that clause $c_i$ is not satisfied  = $(1-p)^{n_p}*p^{n_n}$,
let $1-p < p$. Therefore Prob that it is satisfied is greater than $1-p^{n_p+n_n}$.\\
Therefore the expected number of satisfaction is min(p, $1-p^2$)*($\Sigma w_i - \Sigma v_i$.\\
You can always make weight of positive more thna that of negative literal by renaming.\\
Therefore average is summation $(1-p^2)*w_j$ + $p*w_j$ - $p*v_j$.\\
LP for max sat :\\
Max $\Sigma w_j*z_j$.\\
$\Sigma y_i + \Sigma(1 - y_j) \ge z_j$.\\
$ 0 \le y_i \le 1$ and $ 0 \le z_i \le 1$.\\
$z_j$ is the clause value and $y_i$ are variable values.\\
The LP optimum gives you an upper bound on the integer optimum.\\
View $y_i$ as a probability. Now toss a coin with bias $y_i$
and then assign that value and again compute the LP value after
substituting this in the formula and rewriting the LP and see the value
and go recursively down by conditional probability.
What is the probability C is false ?\\
Prob = $\Pi(1 - {y_j}^*)*\Pi({y_i}^*)$. $1 - y_j$ corresponds to
positive literals might be negated and similarly for $y_i$.\\

By AMGM we know that this is less than or equal to ${(1/l)*(\Sigma 1 - {y_i}^* + \Sigma {y_i}^*)}^l$ which is equal to ${1/l*(l - (\Sigma {y_j}^*  + \Sigma (1 - {y_i}^*))}^l$.\\
From LP condition, we get that this is less than or equal to :\\
${(1/l)*(l - z_i*)}^l$ from which we can get a $1 - 1/e$ of all satisfied clauses.
After this we can go to conditional probability to get a good approx algo.\\

LP based $rendering  \ge \Sigma w_c*{z_c}^*(1-{(l_c-1)/(l_c)}^{l_c}$\\
$P = 1/2 solution \ge \Sigma w_c z_c (1 - 1/2^{l_c}$
$E[max(r_1, r_2)] \ge 1/2(E(r_1) + E(r_2)) \ge 3/4*\Sigma(w_c*)$.\\
If $l_c$ = 1, (1,1/2) === 3/4\\
If $l_c$ = 2, (3/4,3/4) === 3/4\\
If $l_c \geq 3$, (1-1/e,7/8) === 1/2 - 1/2e + 7/16 > 3/4\\
\end{itemize}
\subsection {Prize collecting Steiner Tree}
min $\Sigma c_e x_e + \Sigma (1-y_u)\Pi_u$\\
$\Sigma x_e \ge  y_u, S \subseteq V\{u\}, r \in S$

Threshold based rounding Already done before  \\
Finally we'll have \\
$c(T) \le 2/t \Sigma c_e x_e$\\

We have something like $IntOPT \le 3*LPOpt$
We will show something better than this exists and also the algorithm can help us find it.\\

Select t at random from $[\alpha,1]$, $\alpha$ is some parameter.\\
Probability = $1/(1-\alpha)$\\

$E[penalty] = \Sigma (\Pi_u (1 - y_v^*))/(1-\alpha) integral_{0}^{\alpha} (dt)/(1-t)$
We do the averaging out from $\gamma$ to 1 for the $2/\gamma$ part from which
we get (2/1-$\gamma$)*ln(1/$\gamma$)*$\Sigma c_e*x_e*$.\\

$\Sigma \pi_v$ over $V - T $ where T is the steiner $ tree \le \Sigma
\pi_v$ over all vertices such that ${y_v}^* \le \alpha$ where $\alpha$
is the random variable chosen from $\gamma$ to 1.\\

Now we will bound the penalty of the solution over the average.\\
Now we will split the lost vertices (the one we pay penalty for) into
one interval which is $\alpha \le y_v* \le \gamma$ and $y_v* \le \alpha$.\\
Now using the inequality in the above paragraph, we would get,
it to be $(1/1-\gamma)*\Sigma \pi_v*(1 - y_v*)$.\\
Now with this overall we can say that $\gamma = e^{-1/2}$.\\
Therefore the expected value is $1/(1-e^{-1/2})* LPOpt$.\\
Hence we can find a better algorithm.\\

There is an integrality gap of 2. That is there exists a particualar
graph (descried below) whose LP optimum is atleast twice the ILP
optimum. Consider the case where all edge weights are one and all
node penalties are infinity. Hence the LP would definitely choose
one with no node penalties. Hence if we take any cycle through the vertices
and give all edge weights as 1/2 then we would get n/2 as the value, but
integral optimum is $n-1$.\\

\subsection{3 Colorable}
Input is a graph which is possible to be coloured in three
colours.\\

If a max degree of a vertex is $\delta$ then the maximum you need
is $\delta$ + 1 colours. If a vertex has degree greater than or
equal to $n^{0.5}$, then take the vertex and all their neighbours
and colour it using three colours. This is possible
as the neighbours $must$ be a bipartite graph. Hence we are done here.\\
If the degree of all vertices come with less then we are trivially done.\\
Hence we have a $O(n^{0.5})$ colouring.
A different way of derandomizing max cut is to use
the k-wise independent sample space. It is known
that there exists a $n^2$ size such sample space.\\

If the graph is dense then it might be easier to 3 colour
the graph.\\

Colouring via Dominating set can be done as follows :\\
Take the dominating set on one side S and let the other side have
the rest of the vertices N(S). Therefore if N(S) is list 2
colourable then we are done, after dominating set is coloured with 3
colours. We can solve list 2 colouring via 2 SAT (setup a reduction to it).\\
How large is the dominating set ?\\
Let us assume that the minimum degree of a vertex $\delta*n$.\\
$P = c*ln(n)/(\delta*n)$.\\
We will pick a vertex with this probability in our dominating set.\\
E(S) = c*ln(n)/$\delta$.\\
Therefore we will have a runtime of $3^{c/\delta}*n^2$.\\
$E(|S \cap (N(v) \cup v)|) = P(N(v) \cup u)$.\\
$P(S \le (1 - \epsilon)*u) \le {e^{-\epsilon}/{(1-\epsilon)}^{1 - \epsilon}}^u$.\\
If $E(|S|) = c*ln(n)/\delta$, then with high
probability, $|S| \le 2*ln(n)/\delta$.\\
$Pr(|S \cap (N(v) \cup v)| = 0) \le e^{-p*(\delta*n + 1)}$.\\
The above represents the probability for some vertex.\\
Probability that atleast one vertex is left out is upper bounded by
$n*e^{-c*ln(n)}$.\\
Probability that our set is both small and is a dominating set is
high. We can prove that this can be done with probability
greater than or equal to $1 - n^{-2}$.\\
Using this we can get a $n^{0.5}$ colouring.\\
3 colouring :\\
$n^{\delta}$ coloring.\\
\begin{itemize}
\item Deviation bounds.
\item Small dominating sets in demi graph.
\end{itemize}
E(S) = c*ln(n)/$\delta$.\\
Pr($|S| \ge \alpha*E(x)) \le 1/\alpha$.\\
Pr(S is not a dominating set$) \le n*$Pr($S \cap v \cup N(v) = \phi)$.\\
By our methodology, we get a set with a small fraction of vertices
and a dominating set is quite high.\\

Pr(It is dominating and small) = 1 - (Pr(S is not dom) + Pr(S is small)).\\
We can show that this probability is small enough.\\
The algorithm runs in $3^{10*ln(n)/\delta}*n^2$.\\
Hence there is an RP which solves this class of 3 coloring
problems.\\

\subsection{Vector problem}
Lets try 3 coloring.\\
We will have three vetors 120 degrees apart.\\
min $\Sigma v_i.v_j$ $||v_i|| = 1$.\\
If the graph is three colourable, then the minimization
value will be m*cos(2*$\pi$)/3.\\
The rounding analysis will use the fact that it is 3 colourable.\\

Max Cut now :\\
Assign -1 to negative literals and 1 to positive literals.
x is a unit vector (either -1 or 1 considered as ILP).\\
We need to max $\Sigma (1 - x_i.x_j)/2$.
This is over all i j such that an edge exists.
For every vertex we will assign unit vectors. Now
every part of the sum will give 0 if it is not crossing the cut
and will give 1 if it is crossing the cut.\\

\subsection{Multicommodity flow}
G(V, E) is set of all $S_i$, $t_i$ pairs, $1 \le i \le \gamma$.\\
Quality Minimize : minimize the load of the maximum loaded edge. Here
there are paths from $s_i$ to $t_i$ called $P_i$. The one
which is occurring most is the maximum loadded edge.\\
$\Sigma X_p$ over $p \in P_i$ = 1.\\
$\Sigma x_p \le L$ where L is the load and $x_p$ indicates
all the $X_p$ to which any one edge belongs to.\\
The above has exponential number of constraints.\\
In P time we can output a feasible solution with polynomial
number of greater than 0 values even though the number
of variables are exponential.\\

Assume we have a optimum solution to a LP. Rounding
is a probability distribution to sample from the integer
solutions from LP.\\
Typically you will have a handle on expectation use
this for approximation bounds.\\
Let us say we have the fractional solution to the given problem.\\
Let P be a path sampled from ${x_p}^* = 1$.\\
E[$D_e$] is the expectation of load through the experiment.\\
E[$D_e$] = $\Sigma x_p \le L$ (from the constraint specified above).\\
Pr[$D_e > (1+\delta)*L^*]$.\\
We can say that this is less than $n^2*e^{-\delta^2/3*L^*}$.

\subsection{LP}
min CX , $Ax \ge B$, $x \ge 0$.\\
Describes a closed bounded region polytope.\\
If all the corner points are $made$ to be integers, then we can see that
ILP is solvable in P.\\
We can create such problems though by mentioning the convex hull of
the n points in the sample space of the given problem (Integer sample space).
Now go ahead and draw planes corresponding to the faces.\\
Geometric Algorithms and combinatorial optimization.\\

Designing Separation oracle ($q_1, q_2, .... q_n$), given a query plane, it says
yes if $Ax \ge b$ or it gives a separate it with a hyperplane such that
we can say that this plane puts the point to one side and all the valid
points to the LP on the other side. Given all of the constraints,
we can make an ellipsoid which contains the whole of the valid region.
Every time, I query the center of the ellipse, then so(separation oracle)
will give me a hyperplane separating the two if it is not within.
Now add this as a constraint to the original set of constraints.
This will help us reduce the size of the ellipse.
To be more specific, it will definitely put the
center to the other side of the plane (the side of the plane which contains
the valid points). These things are black boxes. This will give us an ellipse
which is atmost 1/2 the size of the original ellipse. When the
center is in the valid region we stop, else if the volume of ellipsoid
becomes less than the error we can tolerate, then we stop.\\

UGC (unique gates conjecture).\\
Semi definite programs more powerful than LP's.
This is solvable in P time using ellipsoid method.
LP duality and slackness.\\
\subsection{SDP}
min $c^T*X*c$.\\
Given :\\
$A.X \ge b$.\\
X is a positive semidefinite matrix and A.X is the inner product.\\
The above is the formulation of  Semi definite programs.\\


PSD :\\
$\forall x, x^T*A*x \ge 0$, then A is PSD.\\
Symmetric matrices :\\
$A^T = A$.\\
$Ax = \lambda*x$.\\
then :
\begin{itemize}
\item $\lambda$ is real.
$Ax = \lambda*x$.\\
${x^T}^* A^T = \overline{\lambda}*{x^T}^*$.\\
(Let $x^* = {x^T}^*$).\\
$x^* A x = \lambda x^* x$.\\
$x^* A x = \overline{\lambda} x^* x$.\\
Therefore $\lambda = \overline{\lambda}$.\\$Ax = \lambda*x$.\\
Hence it is real.\\
\item There are atmost n distinct $\lambda's$ given that dimension of x is n.\\
The following proves this statement.
\item If $\lambda_i \ne \lambda_j$ then $x_i.x_j = 0$.
\end{itemize}

$A = \Sigma \lambda_i*(x_i)*{(x_i)}^T$.\\
Where $x_i$ are orthogonal vectors which are eigenvectors.\\
We can easily show that
$\forall y, Ay = \Sigma \lambda_i*(x_i)*{(x_i)}^T$.\\
Hence the above claim is true (ie) :\\
$A = \Sigma \lambda_i*(x_i)*{(x_i)}^T$.\\

If $x_1, x_2$ are PSD, then $\alpha*x_1 + (1 - \alpha)*x_2$ is also PSD.\\
Ellipsoid like methods work here.\\

SDP can be solved in P time.\\
Randomized rounding using SDP optimum.\\
Separation oracle usage.\\

Given a set of vertices, we need to paritition the vertices
into sets of -1 and 1.\\
We write $max(\Sigma 1 - y_i*y_j)/2$ which again computes max cut.\\

It can be formulated in SDP or Vector program saying the norm of $||y_i|| = 1$.\\
Any PSD can be writted as $X*X^T$.\\

Relaxation value(y*) is greater than or equal to MAX CUT.\\
We get a cut S whose value is atleast 0.878 times the MAX CUT.\\

f(x) = $(1/{(2*\Pi*\sigma^2)}^0.5)*e^{-{(x-\mu)}^2/2*\sigma^2}$.\\
Variance is $E({(x - \mu)}^2)$.\\

Check that the probability of $r/||r||$ is equal to 1/surface area of sphere.\\
This is also a normal distribution.\\
L = $v_i$ such that $v_i.r < 0$.\\
and U = $v_i$ such that $v_i.r \ge 0$.\\
r is the vector that degenerated the plane that is v such that v.r = 0.\\
The probability that the edge x y is cut = 2*Pr($v_i.r < 0$ and $v_j.r \ge 0$).\\

\begin{itemize}
\item View y* as n vectors $v_1, v_2, ... v_n$.
\item Select an r each coordinate from N(0, 1).
\item Projection of r onto the place containing $v_x$ and $v_y$
is also uniformly distributed on the circle containing $v_x$ and $v_y$.
The probability that r falls in the angle in the required region
of $\theta$ is $2*\theta/2*\pi$ = $\theta/\pi$.\\
Based on the splitting based on this one vector, we can round the
vectors.\\
\item $v_x.v_y = cos(\theta)$.\\
Therefore $\theta/\pi = {cos}^{-1}(v_i.v_j)/\pi \ge 0.878*(1-x)/2$.
\end{itemize}
Linearity of expectation saves us again. Therefore the probability
that an edge is cut is $\Sigma$ (over all e) Pr(e) is cut = $\Sigma {cos}^{-1}(v_x.v_y)/\pi$.\\
which is greater than or equal to 0.878*value(y*) which is greater than or equal to
0.878*MAXCUT.\\

Quadratic programming OPT :\\
max $\Sigma a_{ij}*x_i*x_j$ where $x_i \in {-1, +1}$.\\
max $\Sigma a_{ij}*v_i.v_j$ where $||v_i|| = 1$.\\
We select a random hyperplane and set these variables as 1 and -1
correspondingly from the vector we chose if $r.v_i \ge 0$ then $x_i = 1$ and
-1 otherwise.\\

$E(x_i.x_j) = Pr[x_i.x_j = 1] - Pr[x_i.x_j = -1] = (1 - Pr[x_i.x_j = -1]) - (Pr[x_i.x_j = -1])$.\\
We know that the $Pr[x_i.x_j = - 1] = (1/\pi)*{cos}^{-1}(v_i.v_j)$ that is the
vectors are cut by the random hyper plane. Hence the expectation is $1 - (2/\pi)*{cos}^{-1}(v_i.v_j)$
.\\
If we substitute cos inverse by $\pi/2 - sin$ inverse we get,\\
$E(x_i.x_j) = (2/\pi)*{sin}^{-1}(v_i.v_j)$.\\
Hence if we get that $\Sigma a_{ij}*{sin}^{-1}(v_i.v_j) - \Sigma a_{ij}*v_i.v_j$
is greater than zero then we have a $2/\pi$ approximation.\\

$a_{ij} \ge 0$ only gives us the power which we used in the last lecture.\\
Consider the matrix $X = V^T.V$. Hence X is positive semidefinite.\\
Put down V as the column vectors got by setting $v_i$ that is the
frst column of V is $v_1$ and so on.\\
Now X is PSD. Termwise multiplication of PSD's gives us a PSD.\\
If C is PSD then sum over all elements in C is greater than or
equal to 0.\\
Now we are going to make up a PSD such that the sum over all
elements of the PSD is $\Sigma a_{ij}*{sin}^{-1}(v_i.v_j) - \Sigma a_{ij}*v_i.v_j$ . Let this term be called $\gamma$.\\

Let A = [$a_{ij}$], B = [${sin}^{-1}(v_i.v_j) - (v_i.v_j)]$.
Let us analyze ${sin}^{-1}(x) - x$.\\

The taylor series expansion can be looked at.\\
From some arguments we can get that matrix formed by ${sin}^{-1}(x) - x$ is
PSD.\\
Assume that $A$ is positive semidifinite and we get the above calculation easily.\\
As termmwise multiplication of two PSD, gives a PSD, we can end up with a PSD
and hence $\Sigma a_{ij}*{sin}^{-1}(v_i.v_j) - \Sigma a_{ij}*v_i.v_j$ is greater
than or equal to 0. Hence the $2/\Pi$ approximation holds.\\

We can have bicriteria optimizations like $\alpha*MST$ + $(1-\alpha)*SSSP$.\\
(SSSP refers to single source shortest path).
Colouring combinatorial :\\
Given a planar graph checking whether it is 3 colourable or a
4 colourable is NP complete.\\

Semidefinite Programming for colouring in P time.\\
If 3 coloring if promised, you can do a $n^{0.287}$ coloring in P time.\\
Assume $v_i$ is the vector for vertex i.\\
Observation :\\
$v_i.v_i = 1$.\\
$v_i.v_j \le \lambda$.\\
min $\lambda$.\\
is the SDP we will use. $s = cos(2*\pi/3)$, then we are asured that
the current colouring is a 3 colouring.\\
Pick r random hyperplanes passing through the origin. This
would give us $2^r$ partitioning. r is going to be less than $log(n)$.\\
For each part, assign a different colour.\\
$Pr[c(u_i) = c(u_j)]$ where $u_i, u_j$ is an edge is as follows :\\
$c(u_i) \ne c(u_j) \leftrightarrow \exists$ such a plane which
puts them in opposite regions (lets say some n).\\
$c(u_i) = c(u_j) \leftrightarrow \forall n (v_i.n \& v_j.n < 0)$.
As of now we have not used the three colourableness of the
graph here.\\
Probability that no planes cross an edge ($c(u_i) = c(u_j)$) =
$(1 - {cos}^{-1}(v_i.v_j)/\pi)^r$.\\
${cos}^{-1}(v_i.v_j) \ge {cos}^{-1}(\lambda)$.\\
Because the graph is three colourable, we can be assured that we would
get $\lambda \le -1/2$ (which is atleast cos($2*\pi$/3)).\\
$(1 - {cos}^{-1}(v_i.v_j)/\pi)^r \le (1 - {cos}^{-1}(\lambda)/\pi)^r \le 1/3^r$.\\
Hence we are done.\\

Round by Round : At every iteration choose different hyper planes
and recolour the badly coloured edges (ie) take
out the vertices's colours and recolour the
sub graph.\\
% Markov inequality :\\
% $Pr(X > 2*m/3^r) < 1/2$.\\
% $Pr(\le 2*m/3^r$ edges are badly coloured $) \ge 1/2$.\\
Given a 3 colouring graph, we set m so as to the number
of vertices (which is atmost twice the number of edges badly coloured)
which are badly coloured is less than or equal to $2*m*/3^r$.



\end{document}
